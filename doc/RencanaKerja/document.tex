\documentclass[a4paper,twoside]{article}
\usepackage[T1]{fontenc}
\usepackage[bahasa]{babel}
\usepackage{graphicx}
\usepackage{graphics}
\usepackage{float}
\usepackage[cm]{fullpage}
\pagestyle{myheadings}
\usepackage{etoolbox}
\usepackage{setspace} 
\usepackage{lipsum} 
\setlength{\headsep}{30pt}
\usepackage[inner=2cm,outer=2.5cm,top=2.5cm,bottom=2cm]{geometry} %margin
% \pagestyle{empty}

\makeatletter
\renewcommand{\@maketitle} {\begin{center} {\LARGE \textbf{ \textsc{\@title}} \par} \bigskip {\large \textbf{\textsc{\@author}} }\end{center} }
\renewcommand{\thispagestyle}[1]{}
\markright{\textbf{\textsc{AIF401/AIF402 \textemdash Rencana Kerja Skripsi \textemdash Sem. Ganjil 2021/2022}}}

\newcommand{\HRule}{\rule{\linewidth}{0.4mm}}
\renewcommand{\baselinestretch}{1}
\setlength{\parindent}{0 pt}
\setlength{\parskip}{6 pt}

\onehalfspacing
 
\begin{document}

\title{\@judultopik}
\author{\nama \textendash \@npm} 

%tulis nama dan NPM anda di sini:
\newcommand{\nama}{Rajasa Cikal Maulana Solihin}
\newcommand{\@npm}{2017730084}
\newcommand{\@judultopik}{Pembuatan Ulang Aplikasi WSDC 2017 Bali dengan Ionic 5} % Judul/topik anda
\newcommand{\jumpemb}{1} % Jumlah pembimbing, 1 atau 2
\newcommand{\tanggal}{01/01/1900}

% Dokumen hasil template ini harus dicetak bolak-balik !!!!

\maketitle

\pagenumbering{arabic}

\section{Deskripsi}
WSDC ({\it World Schools Debating Championships}) merupakan sebuah turnamen debat bahasa inggris tahunan untuk tim-tim tingkat sekolah menengah yang mewakili berbagai negara. Pada tahun 1991, kejuaraan diadakan di Edinburgh. Dan sejak saat itu nama {\it World Schools Debating Championships} digunakan dan berlangsung hingga saat ini. Ionic merupakan {\it Software Development Kit} (SDK) {\it open source} yang digunakan untuk pengembangan aplikasi seluler yang bersifat hibrida pada tahun 2013 oleh Max Lynch, Ben Sperry, dan Adam Bradley, dibangun menggunakan AngularJS dan Apache Cordova. Saat ini sudah terdapat sebuah aplikasi WSDC 2017 Bali berbasis {\it framework} Ionic versi 3 yang dibangun pada tahun 2017 untuk mendukung berjalannya acara tersebut. Karena Ionic versi 3 sudah terlalu lawas dan tidak ada pembaruan lagi, maka pada skripsi ini akan dibuat sebuah perangkat lunak WSDC menggunakan {\it framework} Ionic versi 5, sebagai pengembangan dari aplikasi yang sudah ada. 

\section{Rumusan Masalah}
Rumusan masalah yang akan dibahas pada skripsi ini adalah sebagai berikut :
\begin{itemize}
	\item Fitur-fitur apa yang akan tersedia di aplikasi WSDC terbaru?
	\item Bagaimana membangun aplikasi WSDC menggunakan {\it framework} Ionic versi 5?
	\item Bagaimana melakukan migrasi Ionic versi 3 ke Ionic versi 5?
\end{itemize}

\section{Tujuan}
Tujuan yang ingin dicapai dari penulisan skripsi ini adalah sebagai berikut :
\begin{itemize}
	\item Mendefinisikan fitur-fitur yang akan tersedia di aplikas WSDC terbaru.
	\item Membangun aplikasi WSDC menggunakan {\it framework} Ionic versi 5.
	\item Melakukan migrasi Ionic versi 3 ke Ionic versi 5.
\end{itemize}

\section{Deskripsi Perangkat Lunak}
Perangkat lunak yang akan dibuat memiliki fitur-fitur yang sama dengan perangkat lunak terdahulu, namun menggunakan {\it framework} Ionic versi 5 yang terbaru dan tanpa memiliki fitur notifikasi. Perangkat lunak akhir akan memiliki fitur minimal sebagai berikut:
\begin{itemize}
	\item Pengguna dapat berpindah halaman.
	\item Pengguna dapat melihat berita.
	\item Pengguna dapat melihat peta yang menunjukan lokasi tempat dimana berlangsungnya kegiatan.
	\item Pengguna dapat melihat jadwal acara yang secara otomatais menunjukan hari yang aktif.
	\item Pengguna dapat melihat hasil perlombaan.
	\item Fitur lain yang dirasa dibutuhkan setelah analisis kebutuhan akan ditambahkan kemudian.
		
\end{itemize}

\section{Detail Pengerjaan Skripsi}
Bagian-bagian pekerjaan skripsi ini adalah sebagai berikut :
	\begin{enumerate}
		\item Melakukan studi mengenai {\it framework} Ionic versi 3 dan versi 5.
		\item Menganalisis aplikasi WSDC 2017 Bali.
		\item Mempelajari bagaimana cara melakukan migrasi Ionic versi 3 ke versi 5.
		\item Mendesain kelas aplikasi.
		\item Membangun aplikasi WSDC dengan {\it framework} Ionic versi 5. 
		\item Melakukan pengujian dan eksperimen.
		\item Menulis dokumen skripsi.
	\end{enumerate}

\section{Rencana Kerja}
Rincian capaian yang direncanakan di Skripsi 1 adalah sebagai berikut:
\begin{enumerate}
\item Mempelajari {\it framework} Ionic versi 3 dan versi 5.
\item Menganalisis aplikasi WSDC yang sudah ada.
\item Menulis dokumen skripsi (Bab 1-3).
\end{enumerate}

Sedangkan yang akan diselesaikan di Skripsi 2 adalah sebagai berikut:
\begin{enumerate}
\item Melakukan migrasi dari Ionic versi 3 ke versi 5
\item Mengimplementasi {\it framework} Ionic versi 5 ke dalam aplikasi WSDC. 
\item Melakukan pengujian dan eksperimen terhadap aplikasi yang dibuat.
\item Menulis dokumen skripsi (Bab 4-6)
\end{enumerate}

\newpage

\vspace{1cm}
\centering Bandung, \tanggal\\
\vspace{2cm} \nama \\ 
\vspace{1cm}

Menyetujui, \\
\ifdefstring{\jumpemb}{2}{
\vspace{1.5cm}
\begin{centering} Menyetujui,\\ \end{centering} \vspace{0.75cm}
\begin{minipage}[b]{0.45\linewidth}
% \centering Bandung, \makebox[0.5cm]{\hrulefill}/\makebox[0.5cm]{\hrulefill}/2013 \\
\vspace{2cm} Nama: \makebox[3cm]{\hrulefill}\\ Pembimbing Utama
\end{minipage} \hspace{0.5cm}
\begin{minipage}[b]{0.45\linewidth}
% \centering Bandung, \makebox[0.5cm]{\hrulefill}/\makebox[0.5cm]{\hrulefill}/2013\\
\vspace{2cm} Nama: \makebox[3cm]{\hrulefill}\\ Pembimbing Pendamping
\end{minipage}
\vspace{0.5cm}
}{
% \centering Bandung, \makebox[0.5cm]{\hrulefill}/\makebox[0.5cm]{\hrulefill}/2013\\
\vspace{2cm} Nama: \makebox[3cm]{\hrulefill}\\ Pembimbing Tunggal
}
\end{document}

