\documentclass[a4paper,twoside]{article}
\usepackage[T1]{fontenc}
\usepackage[bahasa]{babel}
\usepackage{graphicx}
\usepackage{graphics}
\usepackage{float}
\usepackage[cm]{fullpage}
\pagestyle{myheadings}
\usepackage{etoolbox}
\usepackage{setspace} 
\usepackage{lipsum} 
\setlength{\headsep}{30pt}
\usepackage[inner=2cm,outer=2.5cm,top=2.5cm,bottom=2cm]{geometry} %margin
% \pagestyle{empty}

\makeatletter
\renewcommand{\@maketitle} {\begin{center} {\LARGE \textbf{ \textsc{\@title}} \par} \bigskip {\large \textbf{\textsc{\@author}} }\end{center} }
\renewcommand{\thispagestyle}[1]{}
\markright{\textbf{\textsc{Laporan Perkembangan Pengerjaan Skripsi\textemdash Sem. Genap 2015/2016}}}

\onehalfspacing
 
\begin{document}

\title{\@judultopik}
\author{\nama \textendash \@npm} 

%ISILAH DATA BERIKUT INI:
\newcommand{\nama}{Rajasa Cikal Maulana Solihin}
\newcommand{\@npm}{2017730084}
\newcommand{\tanggal}{13/12/2021} %Tanggal pembuatan dokumen
\newcommand{\@judultopik}{Pembuatan Ulang Aplikasi WSDC 2017 Bali dengan Ionic 5} % Judul/topik anda
\newcommand{\kodetopik}{PAN5192}
\newcommand{\jumpemb}{1} % Jumlah pembimbing, 1 atau 2
\newcommand{\pembA}{Pascal Alfadian Nugroho, S.Kom., M.Comp.}
\newcommand{\pembB}{-}
\newcommand{\semesterPertama}{45 - Ganjil 21/22} % semester pertama kali topik diambil, angka 1 dimulai dari sem Ganjil 96/97
\newcommand{\lamaSkripsi}{1} % Jumlah semester untuk mengerjakan skripsi s.d. dokumen ini dibuat
\newcommand{\kulPertama}{Skripsi 1} % Kuliah dimana topik ini diambil pertama kali
\newcommand{\tipePR}{B} % tipe progress report :
% A : dokumen pendukung untuk pengambilan ke-2 di Skripsi 1
% B : dokumen untuk reviewer pada presentasi dan review Skripsi 1
% C : dokumen pendukung untuk pengambilan ke-2 di Skripsi 2

% Dokumen hasil template ini harus dicetak bolak-balik !!!!

\maketitle

\pagenumbering{arabic}

\section{Data Skripsi} %TIDAK PERLU MENGUBAH BAGIAN INI !!!
Pembimbing utama/tunggal: {\bf \pembA}\\
Pembimbing pendamping: {\bf \pembB}\\
Kode Topik : {\bf \kodetopik}\\
Topik ini sudah dikerjakan selama : {\bf \lamaSkripsi} semester\\
Pengambilan pertama kali topik ini pada : Semester {\bf \semesterPertama} \\
Pengambilan pertama kali topik ini di kuliah : {\bf \kulPertama} \\
Tipe Laporan : {\bf \tipePR} -
\ifdefstring{\tipePR}{A}{
			Dokumen pendukung untuk {\BF pengambilan ke-2 di Skripsi 1} }
		{
		\ifdefstring{\tipePR}{B} {
				Dokumen untuk reviewer pada presentasi dan {\bf review Skripsi 1}}
			{	Dokumen pendukung untuk {\bf pengambilan ke-2 di Skripsi 2}}
		}
		
\section{Latar Belakang}
\textit{World Schools Debating Championships} (WSDC) merupakan sebuah turnamen debat bahasa inggris tahunan untuk tim-tim tingkat sekolah menengah yang mewakili berbagai negara. Pada awalnya, kompetisi universitas dunia akan diselenggarakan di Sydney pada bulan juli 1988. Anggota Federasi Debat Australia menyadari bahwa tidak ada acara serupa untuk siswa sekolah menengah. Namun kejuaraan universitas dunia ini menunjukkan potensi yang sangat besar untuk kompetisi debat internasional yang melibatkan siswa dari seluruh dunia. Pada tahun 1991, kejuaraan diadakan di Edinburgh. Dan sejak saat itu nama World Schools Debating Championships digunakan dan berlangsung hingga saat ini. 

Ionic merupakan sebuah kerangka kerja {\it open source} lintas platform yang memungkinkan untuk mengembangkan aplikasi hibrida yang bekerja pada berbagai macam platform seluler seperti {\it android}, iOS, dan Windows. Ionic memiliki berbagai macam \textit{front-end library} dan \textit{User Interface}(UI) {\it Components} yang digunakan untuk  perancangan aplikasi menggunakan teknologi web seperti HTML, {\it Cascading Style Sheets} CSS, dan Javascript. 

Pada Ionic 5, terdapat beberapa kerangka Javascript yang dapat diimplementasikan menggunakan \textit{framework} Ionic, seperti Angular, React, dan Vue. Angular pada awalnya diciptakan oleh karyawan Google, Misko Hevert dan Adam Abrons pada tahun 2008, yang masih bernama AngularJS dan dikembangkan dalam JavaScript. Pada saat itu sebagian besar situs web menggunakan aplikasi multi-halaman, yaitu ketika pengguna mengklik tautan, maka browser harus mengambil dokumen HTML yang diminta dari server. React adalah \textit{library} JavaScript {\it open source} untuk membangun antarmuka pengguna, dikelola oleh Facebook, dapat digunakan dalam berbagai skenario termasuk aplikasi iOS dan Android. Sedangkan Vue merupakan \textit{framework}  progresif untuk membangun antarmuka pengguna untuk web, yang dapat digunakan baik untuk projek kecil dan untuk {\it Single-Page Applications} (SPAs).

WSDC yang diselenggarakan di Bali, Indonesia pada tahun 2017 memiliki sebuah aplikasi bernama WSDC 2017 Bali yang dikembangkan oleh PT DNArtworks menggunakan \textit{framework} Ionic 3 untuk menunjang acara tersebut. Terdapat beberapa fungsi penting di dalam aplikasi ini, diantaranya adalah jadwal untuk kegiatan peserta, berita tentang acara WSDC yang sedang berlangsung, pemberitahuan mengenai kegiatan acara kepada peserta, informasi lokasi dan penunjuk arah ke lokasi kegiatan acara yang sedang berlangsung, dan notifikasi untuk peserta. 

Aplikasi WSDC 2017 Bali yang dibangun pada tahun 2017 oleh PT DNArtworks menggunakan Ionic versi 3. Sedangkan Ionic versi 3 saat ini sudah tidak mendapat pembaruan lagi. Saat ini Ionic semakin berkembang dan sudah mencapai Ionic versi 5. Maka dari itu, pada skripsi ini akan dibuat sebuah aplikasi pembaruan dari aplikasi WSDC 2017 Bali saat ini, dengan menggunakan \textit{framework} Ionic versi 5. \textit{Framework} yang lebih baru memungkinkan perawatan yang lebih efisien, serta dukungan teknologi yang lebih terbarukan.
\section{Rumusan Masalah}
Rumusan masalah yang akan dibahas pada skripsi ini adalah sebagai berikut :
\begin{itemize}
	\item Fitur-fitur apa yang akan tersedia di aplikasi WSDC terbaru?
	\item Bagaimana membangun aplikasi {\it android} WSDC menggunakan {\it framework} Ionic versi 5?
	\item Bagaimana melakukan migrasi Ionic versi 3 ke Ionic versi 5?
\end{itemize}

\section{Tujuan}
Tujuan yang ingin dicapai dari penulisan skripsi ini adalah sebagai berikut :
\begin{itemize}
	\item Mendefinisikan fitur-fitur yang akan tersedia di aplikasi WSDC terbaru.
	\item Membangun aplikasi {\it android} WSDC menggunakan {\it framework} Ionic versi 5.
	\item Melakukan migrasi Ionic versi 3 ke Ionic versi 5.
\end{itemize}

\section{Detail Perkembangan Pengerjaan Skripsi}
%ada 7, statusnya apa hasilnya apa
Detail bagian pekerjaan skripsi sesuai dengan rencan kerja terkahir adalah sebagai berikut :	
	\begin{enumerate}
		\item \textbf{Melakukan studi mengenai \textit{framework} Ionic versi 3 dan versi 5.} \\
		{\bf Status :} Ada sejak rencana kerja skripsi.\\
		{\bf Hasil :} 
		%Jelasin ada apa aja di ionic 3 sama ionic 5, apa itu ionic, dll
		
		\item \textbf{Menganalisis aplikasi WSDC 2017 Bali.}\\
		{\bf Status :} Ada sejak rencana kerja skripsi.\\
		{\bf Hasil :}Aplikasi WSDC 2017 Bali saat ini menggunakan \textit{framework} Ionic versi 3, dengan Angular versi 4.1.3, dan juga Cordova. Aplikasi WSDC 2017 Bail memiliki berbagai fitur yang dapat diakses oleh pengguna, yaitu halaman utama, melihat berita, pengumuman, jadwal, lokasi acara, hasil pengundian, informasi terkait penyelenggara, serta hasil dari kompetisi WSDC 2017 Bali. 
		
		\item \textbf{Mempelajari bagaimana cara melakukan migrasi Ionic versi 3 ke versi 5.}\\
		{\bf Status :} Ada sejak rencana kerja skripsi.\\
		{\bf Hasil :} Migrasi Ionic versi 3 ke versi 5 dilakukan dengan dua tahap, yaitu pertama melakukan migrasi dari Ionic versi 3 ke Ionic versi 4, lalu yang kedua melakukan migrasi kembali dari Ionic versi 4 ke Ionic versi 5. Pada migrasi pertama, terdapat beberapa langkah, seperti membuat proyek Ionic baru, serta menyalin Angular Services, \textit{root-level items}, global sass, dan bagian-bagian aplikasi. Terdapat perbedaan mengenai komponen yang ada pada Ionic versi 3 dan versi 4, yang kemudian harus disesuaikan. Lalu yang kedua melakukan migrasi kembali dari Ionic versi 4 ke Ionic versi 5, terdapat perbedaan antara kedua versi tersebut. Perbedaan tersebut yaitu terdapat perubahan pada CSS yang digunakan, Events Services, serta pemasangan \textit{package} dan \textit{dependencies}. Selain itu terdapat juga perubahan komponen-komponen yang harus disesuaikan.
		
		\item \textbf{Mendesain kelas aplikasi.}\\
		{\bf Status :} Ada sejak rencana kerja skripsi.\\
		{\bf Hasil :} Belum dikerjakan dan akan dikerjakan pada Skripsi 2.
		
		\item \textbf{Membangun aplikasi WSDC dengan \textit{framework} Ionic versi 5.}\\
		{\bf Status :} Ada sejak rencana kejra skripsi.\\
		{\bf Hasil :} Baru memulai melakukan migrasi aplikasi WSDC 2017 Bali pada tahap pertama, yaitu migrasi dari Ionic versi 3 ke Ionic versi 4. Dengan hasil yaitu aplikasi sudah dapat berjalan, namun belum memiliki fitur apapun selain berpindah halaman. Lalu selanjutnya akan dikerjakan pada Skripsi 2.
		
		\item \textbf{Melakuakn pengujian dan eksperimen.}\\
		{\bf Status :} Ada sejak rencana kejra skripsi.\\
		{\bf Hasil :} Belum dikerjakan dan akan dikerjakan pada Skripsi 2.
		
		\item \textbf{Menulis dokumen skripsi.}\\
		{\bf Status :} Ada sejak rencana kejra skripsi.\\
		{\bf Hasil :} Dokumen skripsi telah telah dikerjakan dengan hasil Bab 1 sampai dengan Bab 3, yang terdiri atas pendahuluan, landasan teori, dan analisis. Untuk pendahuluan, terdiri atas latar belakang, rumusan masalah, tujuan, batasan masalah, metodologi, serta sistematika pembahasan. Lalu untuk landasan teori terdiri atas landasan teori mengenai WSDC 2017 Bali, serta Ionic Framework. Dan yang terakhir yaitu analisis mengenai sistem kini dan sistem usulan.

	\end{enumerate}

\section{Pencapaian Rencana Kerja}
Langkah-langkah kerja yang berhasil diselesaikan dalam Skripsi 1 ini adalah sebagai berikut:
\begin{enumerate}
\item
\item
\item
\end{enumerate}



\section{Kendala yang Dihadapi}
%TULISKAN BAGIAN INI JIKA DOKUMEN ANDA TIPE A ATAU C
Kendala - kendala yang dihadapi selama mengerjakan skripsi :
\begin{itemize}
	\item Terlalu banyak melakukan prokratinasi
	\item Terlalu banyak godaan berupa hiburan (game, film, dll)
	\item Skripsi diambil bersamaan dengan kuliah ASD karena selama 5 semester pertama kuliah tersebut sangat dihindari dan tidak diambil, dan selama 4 semester terakhir kuliah tersebut selalu mendapat nilai E
	\item Mengalami kesulitan pada saat sudah mulai membuat program komputer karena selama ini selalu dibantu teman
\end{itemize}

\vspace{1cm}
\centering Bandung, \tanggal\\
\vspace{2cm} \nama \\ 
\vspace{1cm}

Menyetujui, \\
\ifdefstring{\jumpemb}{2}{
\vspace{1.5cm}
\begin{centering} Menyetujui,\\ \end{centering} \vspace{0.75cm}
\begin{minipage}[b]{0.45\linewidth}
% \centering Bandung, \makebox[0.5cm]{\hrulefill}/\makebox[0.5cm]{\hrulefill}/2013 \\
\vspace{2cm} Nama: \pembA \\ Pembimbing Utama
\end{minipage} \hspace{0.5cm}
\begin{minipage}[b]{0.45\linewidth}
% \centering Bandung, \makebox[0.5cm]{\hrulefill}/\makebox[0.5cm]{\hrulefill}/2013\\
\vspace{2cm} Nama: \pembB \\ Pembimbing Pendamping
\end{minipage}
\vspace{0.5cm}
}{
% \centering Bandung, \makebox[0.5cm]{\hrulefill}/\makebox[0.5cm]{\hrulefill}/2013\\
\vspace{2cm} Nama: \pembA \\ Pembimbing Tunggal
}
\end{document}

