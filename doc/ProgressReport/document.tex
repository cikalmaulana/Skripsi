\documentclass[a4paper,twoside]{article}
\usepackage[T1]{fontenc}
\usepackage[bahasa]{babel}
\usepackage{graphicx}
\usepackage{graphics}
\usepackage{float}
\usepackage[cm]{fullpage}
\pagestyle{myheadings}
\usepackage{etoolbox}
\usepackage{setspace} 
\usepackage{lipsum} 
\setlength{\headsep}{30pt}
\usepackage[inner=2cm,outer=2.5cm,top=2.5cm,bottom=2cm]{geometry} %margin
% \pagestyle{empty}

\makeatletter
\renewcommand{\@maketitle} {\begin{center} {\LARGE \textbf{ \textsc{\@title}} \par} \bigskip {\large \textbf{\textsc{\@author}} }\end{center} }
\renewcommand{\thispagestyle}[1]{}
\markright{\textbf{\textsc{Laporan Perkembangan Pengerjaan Skripsi\textemdash Sem. Genap 2015/2016}}}

\onehalfspacing
 
\begin{document}

\title{\@judultopik}
\author{\nama \textendash \@npm} 

%ISILAH DATA BERIKUT INI:
\newcommand{\nama}{Rajasa Cikal Maulana Solihin}
\newcommand{\@npm}{2017730084}
\newcommand{\tanggal}{21/12/2021} %Tanggal pembuatan dokumen
\newcommand{\@judultopik}{Pembuatan Ulang Aplikasi WSDC 2017 Bali dengan Ionic 5} % Judul/topik anda
\newcommand{\kodetopik}{PAN5192}
\newcommand{\jumpemb}{1} % Jumlah pembimbing, 1 atau 2
\newcommand{\pembA}{Pascal Alfadian Nugroho, S.Kom., M.Comp.}
\newcommand{\pembB}{-}
\newcommand{\semesterPertama}{51 - Ganjil 21/22} % semester pertama kali topik diambil, angka 1 dimulai dari sem Ganjil 96/97
\newcommand{\lamaSkripsi}{1} % Jumlah semester untuk mengerjakan skripsi s.d. dokumen ini dibuat
\newcommand{\kulPertama}{Skripsi 1} % Kuliah dimana topik ini diambil pertama kali
\newcommand{\tipePR}{B} % tipe progress report :
% A : dokumen pendukung untuk pengambilan ke-2 di Skripsi 1
% B : dokumen untuk reviewer pada presentasi dan review Skripsi 1
% C : dokumen pendukung untuk pengambilan ke-2 di Skripsi 2

% Dokumen hasil template ini harus dicetak bolak-balik !!!!

\maketitle

\pagenumbering{arabic}

\section{Data Skripsi} %TIDAK PERLU MENGUBAH BAGIAN INI !!!
Pembimbing utama/tunggal: {\bf \pembA}\\
Pembimbing pendamping: {\bf \pembB}\\
Kode Topik : {\bf \kodetopik}\\
Topik ini sudah dikerjakan selama : {\bf \lamaSkripsi} semester\\
Pengambilan pertama kali topik ini pada : Semester {\bf \semesterPertama} \\
Pengambilan pertama kali topik ini di kuliah : {\bf \kulPertama} \\
Tipe Laporan : {\bf \tipePR} -
\ifdefstring{\tipePR}{A}{
			Dokumen pendukung untuk {\BF pengambilan ke-2 di Skripsi 1} }
		{
		\ifdefstring{\tipePR}{B} {
				Dokumen untuk reviewer pada presentasi dan {\bf review Skripsi 1}}
			{	Dokumen pendukung untuk {\bf pengambilan ke-2 di Skripsi 2}}
		}
		
\section{Latar Belakang}
\textit{World Schools Debating Championships} (WSDC) merupakan sebuah turnamen debat bahasa inggris tahunan untuk tim-tim tingkat sekolah menengah yang mewakili berbagai negara. WSDC pada tahun 2017 diselenggarakan di Bali, Indonesia. Untuk menunjang acara tersebut, terdapat sebuah aplikasi bernama WSDC 2017 Bali. Aplikasi ini memiliki fitur-fitur yang dapat digunakan oleh pengguna, yaitu : 

\begin{enumerate}
	\item Halaman Utama yang berisi berita acara serta pemberitahuan terakhir acara WSDC 2017 Bali.
	\item \textit{Newsletters} yang digunakan untuk melihat berita acara WSDC 2017 Bali.
	\item Halaman \textit{Announcements} untuk melihat pengumuman mengenai acara WSDC 2017 Bali.
	\item Halaman \textit{Schedule} untuk melihat jadwal acara WSDC 2017 Bali.
	\item Halaman \textit{Venues} yang digunakan untuk melihat lokasi dari acara WSDC 2017 Bali.
	\item Halaman \textit{Draw} untuk melihat pembagian \textit{venue} serta pembagian kubu proposisi dan oposisi dari hasil pengundian untuk para negara peserta WSDC 2017 Bali.
	\item Halaman \textit{Result} untuk melihat daftar pemenang dari kompetisi WSDC 2017 Bali. 
	\item Halaman Info untuk melihat kontak-kontak penting yang dapat dihubungi, kata-kata dalam bahasa Indonesia, serta kredit terhadap pembuat dari aplikasi WSDC 2017 Bali.
\end{enumerate}

Aplikasi ini dikembangkan oleh PT DNArtworks menggunakan \textit{framework} Ionic versi 3. Ionic Framework merupakan sebuah kerangka kerja {\it open source} lintas platform yang memungkinkan untuk mengembangkan aplikasi hibrida yang bekerja pada berbagai macam platform seluler seperti {\it android}, iOS, dan Windows. Ionic memiliki berbagai macam \textit{front-end library} dan \textit{User Interface}(UI) {\it Components} yang digunakan untuk  perancangan aplikasi menggunakan teknologi web seperti HTML, {\it Cascading Style Sheets} CSS, dan Javascript. 

Aplikasi WSDC 2017 Bali dibangun menggunakan Ionic versi 3. Sedangkan Ionic versi 3 saat ini sudah tidak mendapat pembaruan lagi. Saat ini Ionic semakin berkembang dan sudah mencapai Ionic versi 5. Maka dari itu, pada skripsi ini akan dibuat sebuah aplikasi pembaruan dari aplikasi WSDC 2017 Bali, dengan menggunakan \textit{framework} Ionic versi 5. Dengan menggunakan \textit{framework} yang lebih baru, maka memungkinkan perawatan yang lebih efisien, serta dukungan teknologi yang lebih terbarukan.
\section{Rumusan Masalah}
Rumusan masalah yang akan dibahas pada skripsi ini adalah sebagai berikut :
\begin{itemize}
	\item Fitur-fitur apa yang akan tersedia di aplikasi WSDC terbaru?
	\item Bagaimana membangun aplikasi {\it android} WSDC menggunakan {\it framework} Ionic versi 5?
	\item Bagaimana melakukan migrasi Ionic versi 3 ke Ionic versi 5?
\end{itemize}

\section{Tujuan}
Tujuan yang ingin dicapai dari penulisan skripsi ini adalah sebagai berikut :
\begin{itemize}
	\item Mendefinisikan fitur-fitur yang akan tersedia di aplikasi WSDC terbaru.
	\item Membangun aplikasi {\it android} WSDC menggunakan {\it framework} Ionic versi 5.
	\item Melakukan migrasi Ionic versi 3 ke Ionic versi 5.
\end{itemize}

\section{Detail Perkembangan Pengerjaan Skripsi}
%ada 7, statusnya apa hasilnya apa
Detail bagian pekerjaan skripsi sesuai dengan rencan kerja terkahir adalah sebagai berikut :	
	\begin{enumerate}
		\item \textbf{Melakukan studi mengenai \textit{framework} Ionic versi 3 dan versi 5.} \\
		{\bf Status :} Ada sejak rencana kerja skripsi.\\
		{\bf Hasil :} Hasil studi mengenai \textit{framework} Ionic versi 3 dan versi 5, sebagaimana yang telah didokumentasikan pada Bab 2 Landasan Teori, bahwa Ionic Framework merupakan sebuah kerangka kerja \textit{open source} lintas platform yang memungkinkan untuk mengembangkan aplikasi hibrida yang bekerja pada berbagai macam platform seluler seperti Android, iOS, dan Windows. Selain itu, telah dipelajari juga bahwa Ionic Framework mendukung komunikasi dengan menggunakan Native API, untuk menambahkan fungsionalitas ke dalam aplikasi Ionic apapun dengan menggunakan Capacitor atau Cordova. Capacitor sendiri merupakan penyedia akses ke perangkat \textit{native} dan fitur platform, serta untuk menyediakan satu set API untuk mengembangkan aplikasi seluler secara hibrida. Sedangkan Cordova, merupakan \textit{framework open source} yang dapat menggunakan teknologi web seperti HTML, JavaScript, dan CSS untuk membangun aplikasi untuk perangkat bergerak yang berjalan pada sistem operasi mobile. Cordova ini lah yang digunakan dalam pengerjaan skripsi ini. Lalu, baik pada Ionic versi 3 maupun versi 5, terdapat UI Component yang digunakan untuk membuat UI untuk aplikasi di dalam Ionic Framework.
		
		\textit{Framework} Ionic versi 3 dan versi 5 telah dipelajari untuk diimplementasikan pada migrasi aplikasi WSDC 2017 Bali menggunakan Ionic versi 5. Perbedaan dari Ionic versi 3 dan versi 5 telah didokumentasikan di dalam Bab 2 Landasan Teori.
		
		\item \textbf{Menganalisis aplikasi WSDC 2017 Bali.}\\
		{\bf Status :} Ada sejak rencana kerja skripsi.\\
		{\bf Hasil :} Dari hasil analisis, diketahui bahwa aplikasi WSDC 2017 Bali saat ini menggunakan \textit{framework} Ionic versi 3, dengan Angular versi 4.1.3, dan juga Cordova. Aplikasi WSDC 2017 Bail memiliki berbagai fitur yang dapat diakses oleh pengguna, yaitu halaman utama, melihat berita, pengumuman, jadwal, lokasi acara, hasil pengundian, informasi terkait penyelenggara, serta hasil dari kompetisi WSDC 2017 Bali. 
		
		\item \textbf{Mempelajari bagaimana cara melakukan migrasi Ionic versi 3 ke versi 5.}\\
		{\bf Status :} Ada sejak rencana kerja skripsi.\\
		{\bf Hasil :} Migrasi Ionic versi 3 ke versi 5 dilakukan dengan dua tahap, yaitu pertama melakukan migrasi dari Ionic versi 3 ke Ionic versi 4, lalu yang kedua melakukan migrasi kembali dari Ionic versi 4 ke Ionic versi 5. Pada migrasi pertama, terdapat beberapa langkah, diantaranya adalah sebagai berikut :
		\begin{enumerate}
			\item Membuat Proyek Ionic Baru, yaitu dengan mengetikan perintah \textbf{ionic start myApp blank} pada \textit{Command Line}.
			\item Menyalin Angular Services yang berada di \textbf{src/providers} pada Ionic 3, menjadi \textbf{src/app/services} pada Ionic 4.
			\item Menyalin \textit{Root-level Items} pada Ionic 3, dengan alamat direktori yang sama atau dengan beberapa perubahan pada Ionic 4. Perubahan-perubahan yang terjadi sudah didokumentasikan pada dokumen skripsi di Bab 2 Landasan Teori, Sub Bab 2.2.3 Migrasi Ionic 3 ke Ionic 5.
			\item Menyalin Global Scss, yang semula berada di \textbf{src/app/global.scss} menjadi \textbf{src/global.scss} pada Ionic 4.
			\item Menyalin Bagian-Bagian Aplikasi, yaitu keseluruhan bagian yang ada pada aplikasi, baik itu halaman maupun fitur yang ada, termasuk juga perubahan UI Component.
		\end{enumerate}
		 
		Lalu yang kedua melakukan migrasi kembali dari Ionic versi 4 ke Ionic versi 5, terdapat perbedaan antara kedua versi tersebut. Perbedaan tersebut antara lain sebagai berikut :
		\begin{enumerate}
			\item CSS, yang terdiri dari CSS Utilities, Display Classes, Kelas Activated, Focuses, dan Hover States, serta Distributed Scss.
			\item Komponen pada Ioinc 5 mengalami perbedaan-perbedaan dibandingkan dengan Ionic 4 yang menyangkut perbedaan nama, penambahan kelas, serta penghapusan atau penggantian komponen.
			\item Warna pada Ionic 5 mengalami perubahan terkait dengan warna bawaan.
			\item Events \textbf{@ionic/angular} dihapus pada Ionic 5.
			\item Package dan Dependencies ditambahkan pada Ionic 5.
		\end{enumerate}				
		Langkah-langkah untuk melakukan migrasi dari Ionic versi 3 ke versi 5 telah dijelaskan pada Bab 2 Landasan Teori, pada Sub Bab 2.2.3 Migrasi Ionic 3 ke Ionic 5.
		%rinciin juga gimana migrasi3 ke 4, 4 ke 5, gausah detail, intinya aja yg udah dipelajari
		
		\item \textbf{Mendesain kelas aplikasi.}\\
		{\bf Status :} Tidak dikerjakan.\\
		{\bf Hasil :} Setelah melakukan studi mengenai Ionic Framework, kelas aplikasi telah dibuat secara otomatis oleh \textit{framework} Ionic. Maka dari itu, kelas aplikasi tidak perlu dibuat secara manual.
		
		\item \textbf{Membangun aplikasi WSDC dengan \textit{framework} Ionic versi 5.}\\
		{\bf Status :} Ada sejak rencana kerja skripsi.\\
		{\bf Hasil :} Baru memulai melakukan migrasi aplikasi WSDC 2017 Bali pada tahap pertama, yaitu migrasi dari Ionic versi 3 ke Ionic versi 4, sesuai dengan yang tertulis pada Bab 2 Landasan Teori. Dengan hasil yaitu aplikasi sudah dapat berjalan, namun belum memiliki fitur apapun selain berpindah halaman, dan menampilkan halaman utama yang hanya berisi satu buah kompnen Card. Lalu selanjutnya akan dikerjakan pada Skripsi 2.
		
		\item \textbf{Melakukan pengujian dan eksperimen.}\\
		{\bf Status :} Ada sejak rencana kerja skripsi.\\
		{\bf Hasil :} Belum dikerjakan dan akan dikerjakan pada Skripsi 2.
		
		\item \textbf{Menulis dokumen skripsi.}\\
		{\bf Status :} Ada sejak rencana kerja skripsi.\\
		{\bf Hasil :} Dokumen skripsi telah telah dikerjakan dengan hasil Bab 1 sampai dengan Bab 3, yang terdiri atas pendahuluan, landasan teori, dan analisis. Untuk pendahuluan, terdiri atas latar belakang, rumusan masalah, tujuan, batasan masalah, metodologi, serta sistematika pembahasan. Lalu untuk landasan teori terdiri atas landasan teori mengenai WSDC 2017 Bali, serta Ionic Framework. Dan yang terakhir yaitu analisis mengenai sistem kini dan sistem usulan.

	\end{enumerate}

\section{Pencapaian Rencana Kerja}
Langkah-langkah kerja yang berhasil diselesaikan dalam Skripsi 1 ini adalah sebagai berikut:
\begin{enumerate}
\item Mempelajari \textit{framework} Ionic versi 3 dan versi 5.
\item Menganalisis aplikasi WSDC 2017 Bali yang sudah ada.
\item Menulis dokumen skripsi untuk Bab 1 sampai dengan Bab 3
\end{enumerate}



\section{Kendala yang Dihadapi}
%TULISKAN BAGIAN INI JIKA DOKUMEN ANDA TIPE A ATAU C
Kendala - kendala yang dihadapi selama mengerjakan skripsi :
\begin{itemize}
	\item Terbatasnya referensi mengenai \textit{framework} Ionic, khusunya yang membahas Ionic versi 5.
\end{itemize}

\vspace{1cm}
\centering Bandung, \tanggal\\
\vspace{2cm} \nama \\ 
\vspace{1cm}

Menyetujui, \\
\ifdefstring{\jumpemb}{2}{
\vspace{1.5cm}
\begin{centering} Menyetujui,\\ \end{centering} \vspace{0.75cm}
\begin{minipage}[b]{0.45\linewidth}
% \centering Bandung, \makebox[0.5cm]{\hrulefill}/\makebox[0.5cm]{\hrulefill}/2013 \\
\vspace{2cm} Nama: \pembA \\ Pembimbing Utama
\end{minipage} \hspace{0.5cm}
\begin{minipage}[b]{0.45\linewidth}
% \centering Bandung, \makebox[0.5cm]{\hrulefill}/\makebox[0.5cm]{\hrulefill}/2013\\
\vspace{2cm} Nama: \pembB \\ Pembimbing Pendamping
\end{minipage}
\vspace{0.5cm}
}{
% \centering Bandung, \makebox[0.5cm]{\hrulefill}/\makebox[0.5cm]{\hrulefill}/2013\\
\vspace{2cm} Nama: \pembA \\ Pembimbing Tunggal
}
\end{document}

