%versi 3 (18-12-2016)
\chapter{Kode Program}
\label{lamp:A}

\section{Komponen Announcements} 
\label{sec:lampiranKomponenAnnouncements}

\begin{lstlisting}[language=JavaScript, label={lst:announcements.page.ts}, caption=annoncement.page.ts]
import { HttpClient } from '@angular/common/http';
import { Component, OnInit } from '@angular/core';
import { Storage } from '@ionic/storage';
import { ToastController } from '@ionic/angular';

@Component({
  selector: 'app-announcement',
  templateUrl: './announcement.page.html',
  styleUrls: ['./announcement.page.scss'],
})

export class AnnouncementPage implements OnInit {
  announcements: Array<{ localtime: string, message: string }>;

  constructor(private http: HttpClient,private storage: Storage,public toastController: ToastController) { }

  ngOnInit(): void {
    this.storage.get('wsdcDataStorage').then((data) => {
      this.announcements = data.announcements;
    })
  }

  doRefresh(refresher) {
    console.log('Begin async operation');
    this.http.get('https://wsdc.dnartworks.com/wsdc_data.json').subscribe(
      (data: any) => {
      this.storage.clear();
      this.storage.set('wsdcDataStorage',data);
      this.announcements = data.announcements;
      console.log("Data updated!"); 
      if (refresher != 0)
          refresher.target.complete();
      },
      (error) => {
        this.presentConnectionAlert();
        refresher.target.complete();
        console.log('error in HTTPRequest JSON');
    });
    setTimeout(() => {
      console.log('Async operation has ended');
      refresher.target.complete();
    }, 30000);
  }

  async presentConnectionAlert() {
    let toast = await this.toastController.create({
      message: 'Failed to refresh information',
      duration: 3000
    });
    toast.present();
  }

  formatDatetime(sqlDatetime: string) {
    var date = new Date(sqlDatetime.substring(0, 10));
    var dayNames = ["Sunday", "Monday", "Tuesday", "Wednesday", "Thursday", "Friday", "Saturday"];
    var monthNames = ["January", "February", "March", "April", "May", "June", "July", "August", "September", "October", "November", "December"];
    var dayOfWeek = date.getDay();
    var day = date.getDate();
    var monthIndex = date.getMonth();
    var time=sqlDatetime.substring(16,4)
    return time.substring(7) + ' | ' + dayNames[dayOfWeek] + ', ' + day + ' ' + monthNames[monthIndex];
  }
}

\end{lstlisting} 

\begin{lstlisting}[language=html, label={lst:announcements.page.html}, caption=annoncement.page.html]
<ion-header>
  <ion-toolbar>
    <ion-buttons slot="start">
      <ion-menu-button></ion-menu-button>
    </ion-buttons>
    <ion-title>Announcements</ion-title>
  </ion-toolbar>
</ion-header>

<ion-content>
  <ion-refresher slot="fixed" (ionRefresh)="doRefresh($event)">
    <ion-refresher-content></ion-refresher-content>
  </ion-refresher>
  <ion-list>
    <ion-item color="wsdc-blue" *ngFor="let announcement of announcements; let i = index">
      <ion-label class="ion-text-wrap">
        <h3>{{ formatDatetime(announcement.localtime) }}</h3>
        <p>{{ announcement.message }}</p>
      </ion-label>
    </ion-item>
  </ion-list>
</ion-content>
\end{lstlisting} 

\begin{lstlisting}[language=JavaScript, label={lst:draw.page.ts}, caption=draw.page.ts]
import { Component, ElementRef, OnInit, ViewChild } from '@angular/core';
import { LoadingController } from '@ionic/angular';
import { Storage } from '@ionic/storage';

@Component({
  selector: 'app-draw',
  templateUrl: './draw.page.html',
  styleUrls: ['./draw.page.scss'],
})

export class DrawPage implements OnInit {
  @ViewChild('drawIFrame') drawIFrame: ElementRef;
  loading: any;
  constructor(private storage: Storage,public loadingController: LoadingController) { }

  ngOnInit() {
    this.storage.get('wsdcDataStorage').then((data) => {
      this.drawIFrame.nativeElement.contentWindow.location.assign(data.draws);
    });
    this.presentLoading();
  }

  async presentLoading() {
    this.loading = await this.loadingController.create({
      message: 'Please wait...',
      backdropDismiss: true // If true, the loading indicator will be dismissed when the backdrop is clicked.
    });
    this.loading.present();
    setTimeout(() => {
      this.loading.dismiss();
    }, 1000);
  }

  onDrawIframeLoad(){
    let doc = this.drawIFrame.nativeElement.contentWindow.document;
    let elements = [
      doc.getElementById('header'),
      doc.getElementById('page_header'),
      doc.getElementById('footer')
    ];
    elements.forEach(function (element) {
      if (element) {
        element.style.display = 'none';
      }
    })
    this.drawIFrame.nativeElement.style.display = 'block';
    console.log(this.loading);
    
  }
}
\end{lstlisting} 

\begin{lstlisting}[language=html, label={lst:draw.page.html}, caption=draw.page.html]
<ion-header>
  <ion-toolbar>
    <ion-buttons slot="start">
      <ion-menu-button></ion-menu-button>
    </ion-buttons>
    <ion-title>Draw</ion-title>
  </ion-toolbar>
</ion-header>

<ion-content>
  <iframe #drawIFrame (load)="onDrawIframeLoad()"></iframe>
</ion-content>
\end{lstlisting} 

\begin{lstlisting}[language=JavaScript, label={lst:home.page.ts}, caption=home.page.ts]
import { Component, OnInit } from '@angular/core';
import { HttpClient } from '@angular/common/http';
import { Browser } from '@capacitor/browser';
import { Storage } from '@ionic/storage';
import { Router } from '@angular/router';
import { SplashScreen } from '@capacitor/splash-screen';
import { ToastController } from '@ionic/angular';

@Component({
  selector: 'app-home',
  templateUrl: './home.page.html',
  styleUrls: ['./home.page.scss'],
})

export class HomePage implements OnInit {
  wsdcData:any;

  constructor(private http: HttpClient,private storage: Storage,private router: Router,public toastController: ToastController) { }

  ionViewDidEnter(){ //runs when the page has fully entered and is now the active page.
    SplashScreen.hide()
  }

  ngOnInit(): void { //called after Angular has initialized all data-bound properties of a directive
    this.storage.get('wsdcDataStorage').then((data) => {
      
      if(data == null){
        //Default from asset
        this.http.get('../assets/json/wsdc_data.json').subscribe((data: any) => {
          this.wsdcData = data;
          this.storage.set('wsdcDataStorage',data);       
        },
        error => {
          this.showToast('Failed to refresh information from local storage');
        });
      }else{    
        this.wsdcData = data;
      }
      
      // Refresh data
      setTimeout(() => {
        this.http.get('http://wsdc.dnartworks.com/wsdc_data.json')
        .subscribe((data: any) => {
          this.storage.set('wsdcDataStorage', data);
          this.wsdcData = data;
        },
        error => {
          this.showToast('Failed to refresh information');
        });
      }, 1000);
    })
    
  }

  doRefresh(refresher) {
    this.http.get('https://wsdc.dnartworks.com/wsdc_data.json').subscribe((data: any) => {
      this.storage.clear();
      this.storage.set('wsdcDataStorage',data);
      this.wsdcData = data;
    },
    error => {
      // Timeout or no connection
      this.showToast('Failed to refresh information');
      refresher.complete();
    });

    setTimeout(() => {
      refresher.target.complete();
    }, 2000);
  }

  async showToast(message: string, duration: number=3000) {
    let toast = await this.toastController.create({
      message: message,
      duration: duration
    });
    toast.present();
  }

  formatDatetime(sqlDatetime: string) {
    if (sqlDatetime === null) {
      return null;
    }
    var time=sqlDatetime.substring(16,4)
    var date = new Date(sqlDatetime.substring(0, 10));
    var dayNames = ["Sunday", "Monday", "Tuesday", "Wednesday", "Thursday", "Friday", "Saturday"];
    var dayOfWeek = date.getDay();
    return dayNames[dayOfWeek] + ', ' + time.substring(7);
  }

  // ionic 6 : use capacitor in app browser
  launch(newsUrl: string) {
    Browser.open({ url: newsUrl });
  }

  // ionic 6 : use ionic angular router for change page
  onAnnouncementClick() {
    this.router.navigate(['announcement']);
  }
}

\end{lstlisting} 

\begin{lstlisting}[language=html, label={lst:home.page.html}, caption=home.page.html]
<ion-header>
  <ion-toolbar>
    <ion-buttons slot="start">
      <ion-menu-button></ion-menu-button>
    </ion-buttons>
    <ion-title>Home</ion-title>
  </ion-toolbar>
</ion-header>

<ion-content>
  <ion-refresher slot="fixed" (ionRefresh)="doRefresh($event)">
    <ion-refresher-content></ion-refresher-content>
  </ion-refresher>
  <ion-card (click)="onAnnouncementClick()">
    <ion-grid>
      <ion-row>
        <ion-col size="9">
          <ion-card-header>
            <ion-card-title>Latest Announcement</ion-card-title>
          </ion-card-header>
        
          <ion-card-content>
              <h3>{{ formatDatetime(wsdcData?.announcements[0].localtime) }}</h3>
              <p>{{ wsdcData?.announcements[0].message }}</p>
          </ion-card-content>
        </ion-col>
        <ion-col size="3">
          <img src="assets/icon/announcement.png"/>
        </ion-col>
      </ion-row>
    </ion-grid>
  </ion-card>

  <ion-list>
    <ion-list-header>
      <ion-icon name="book-outline"></ion-icon>
      <ion-label class="label_newsletters"><h1><b> Newsletters</b></h1></ion-label>
    </ion-list-header>
    <ion-item *ngFor="let wsdcNews of wsdcData?.newsletters;">
      <div class="newsletters" >
        <img src="http://wsdc.dnartworks.com/uploads/newsletter/{{wsdcNews.id}}/thumbnail.jpg">
        <h2 text-wrap>{{wsdcNews.title}}</h2> <br>
        <ion-button class="ion-padding-end" full block color="danger" (click)="launch(wsdcNews.url)">Read More</ion-button>
      </div>
    </ion-item>
  </ion-list>
</ion-content>
\end{lstlisting} 

\begin{lstlisting}[language=JavaScript, label={lst:info.page.ts}, caption=info.page.ts]
import { Component, ViewEncapsulation  } from '@angular/core';
import { Storage } from '@ionic/storage';

@Component({
  selector: 'app-info',
  templateUrl: './info.page.html',
  styleUrls: ['./info.page.scss'],
  encapsulation: ViewEncapsulation.None,
})

export class InfoPage{
  wsdcInfoData: any;

  constructor(private storage: Storage) { 
    
    this.storage.get('wsdcDataStorage').then((data) => {
      this.wsdcInfoData = data.info;
      this.wsdcInfoData = this.wsdcInfoData.replace(new RegExp('icon-telephone','g'), '<img src="assets/icon/telephone.png" alt="Telephone Icon" class="icon"/>');
    })
  }

}

\end{lstlisting} 

\begin{lstlisting}[language=html, label={lst:info.page.html}, caption=info.page.html]
<ion-header>
  <ion-toolbar>
    <ion-buttons slot="start">
      <ion-menu-button></ion-menu-button>
    </ion-buttons>
    <ion-title>Info</ion-title>
  </ion-toolbar>
</ion-header>

<ion-content>
  <ion-grid>
    <ion-row>
      <div [innerHTML]="wsdcInfoData"></div>
    </ion-row>
  </ion-grid>
</ion-content>

\end{lstlisting}

\begin{lstlisting}[language=JavaScript, label={lst:result.page.ts}, caption=result.page.ts]
import { Component, ElementRef, OnInit, ViewChild } from '@angular/core';
import { Storage } from '@ionic/storage';
import { LoadingController } from '@ionic/angular';

@Component({
  selector: 'app-result',
  templateUrl: './result.page.html',
  styleUrls: ['./result.page.scss'],
})

export class ResultPage implements OnInit {
  @ViewChild('resultIFrame') resultIFrame: ElementRef;

  constructor(private storage: Storage,public loadingController: LoadingController) { }

  ngOnInit() {
    this.storage.get('wsdcDataStorage').then((data) => {
      this.resultIFrame.nativeElement.contentWindow.location.assign(data.results);
    });
    this.presentLoading();
  }

  async presentLoading() {
    const loading = await this.loadingController.create({
      message: 'Please wait...',
      backdropDismiss: true
    });

    await loading.present();

    setTimeout(() => {
      loading.dismiss();
    }, 500);
  }

  onResultIframeLoad(){
    let doc = this.resultIFrame.nativeElement.contentWindow.document;
    let elements = [
      doc.getElementById('header'),
      doc.getElementById('page_header'),
      doc.getElementById('footer')
    ];
    elements.forEach(function (element) {
      if (element) {
        element.style.display = 'none';
      }
    })
    this.resultIFrame.nativeElement.style.display = 'block';
  }

}

\end{lstlisting} 

\begin{lstlisting}[language=html, label={lst:result.page.html}, caption=result.page.html]
<ion-header>
  <ion-toolbar>
    <ion-buttons slot="start">
      <ion-menu-button></ion-menu-button>
    </ion-buttons>
    <ion-title>Result</ion-title>
  </ion-toolbar>
</ion-header>

<ion-content>
  <iframe #resultIFrame (load)="onResultIframeLoad()"></iframe>
</ion-content>

\end{lstlisting}

\begin{lstlisting}[language=JavaScript, label={lst:schedule.page.ts}, caption=schedule.page.ts]
import { Component, ElementRef} from '@angular/core';
import { ViewChild } from '@angular/core';
import { IonSlides } from '@ionic/angular';
import { Storage } from '@ionic/storage';

@Component({
  selector: 'app-schedule',
  templateUrl: './schedule.page.html',
  styleUrls: ['./schedule.page.scss'],
})
export class SchedulePage{
  @ViewChild('scheduleSlider', { static: false }) slider: IonSlides;
  @ViewChild('segmentContainer', { static: false }) segmentContainer: ElementRef;
  schedules: any;
  slideOpts = {
    initialSlide: 0,
    speed: 400,
  };
  selectedSegmentIdx: any;
  currentIndex:any;

  constructor(private storage: Storage) {
    
    this.storage.get('wsdcDataStorage').then((val) => {
      this.schedules = val.schedules;
      this.selectedSegmentIdx = 0;
    });
  }

  getDayName(sqlDate: string) {
    var date = new Date(sqlDate);
    var dayNames = ["Sun", "Mon", "Tue", "Wed", "Thu", "Fri", "Sat"];
    var dayOfWeek = date.getDay();
    return dayNames[dayOfWeek];
  }

  getDate(sqlDate: string) {
    var date = new Date(sqlDate);
    return date.getDate();
  }

  onSlideChanged() {
    this.slider.getActiveIndex().then((index: number) => {
      this.currentIndex = index;
      document.getElementById(this.currentIndex).scrollIntoView({
        behavior: 'smooth',
        block: 'center',
        inline: 'center'
      });
      this.selectedSegmentIdx = index;
    });
  }

  onSegmentChanged(segmentButton) {
    this.slider.slideTo(segmentButton.detail.value);
  }
}

\end{lstlisting} 

\begin{lstlisting}[language=html, label={lst:schedule.page.html}, caption=schedule.page.html]
<ion-header>
  <ion-toolbar>
    <ion-buttons slot="start">
      <ion-menu-button></ion-menu-button>
    </ion-buttons>
    <ion-title>Schedule</ion-title>
  </ion-toolbar>
</ion-header>

<ion-content>
  <div id="schedulesContainer">
    <div id="schedulesSegments" slot="fixed">
      <ion-segment #segmentContainer scrollable [(ngModel)]="selectedSegmentIdx"  (ionChange)="onSegmentChanged($event)"> 
        <ion-segment-button *ngFor="let schedule of schedules; let i = index;" [value]="i" [id]="i">
          <ion-label>
            <div class="day">{{ getDayName(schedule.date)}}</div>
            <div class="date">{{ getDate(schedule.date) }}</div>
          </ion-label>
        </ion-segment-button>
      </ion-segment>
    </div>
    <div id="schedulesSlides">
      <ion-slides #scheduleSlider [options]="slideOpts" (ionSlideDidChange)="onSlideChanged()">
        <ion-slide *ngFor="let schedule of schedules;">
          <ion-list>
            <ion-item *ngFor="let agenda of schedule.agenda;" >
              <ion-note item-start>
                  {{agenda.start}} <br> 
                  {{agenda.end}}
              </ion-note>
              <ion-label class="ion-text-wrap">
                <h3>{{agenda.title}}</h3>
                <p>{{agenda.subtitle}}</p>
              </ion-label>
            </ion-item>
          </ion-list>
        </ion-slide>
      </ion-slides>
    </div>
  </div>
</ion-content>

\end{lstlisting}

\begin{lstlisting}[language=JavaScript, label={lst:venues.page.ts}, caption=venues.page.ts]
import { Component, OnInit } from '@angular/core';
import { Router, NavigationExtras } from '@angular/router';
import { Storage } from '@ionic/storage';

@Component({
  selector: 'app-venues',
  templateUrl: './venues.page.html',
  styleUrls: ['./venues.page.scss'],
})
export class VenuesPage{
  venuesData: Array<{ id: string, name: string, icon: string, geojson: any, colorIdx: number }>;
  valVenues: any;
  constructor(private router: Router,private storage: Storage) {
    this.storage.get('wsdcDataStorage').then((val) => {
      this.valVenues = val.venues;
      this.venuesData = [];
      let currColorIdx: number = 1;
      for (let venue of this.valVenues) {
        this.venuesData.push({
          id: venue.id,
          name: venue.name,
          icon: venue.icon,
          geojson: venue.geojson,
          colorIdx: currColorIdx,
        });
        if (currColorIdx > 4) {
          currColorIdx = 1;
        } else {
          currColorIdx++;
        }
      }
    })
  }

  itemTapped(wsdcVenue) {
    // this.router.navigate(['venues-map', id])
    let navigationExtras: NavigationExtras = {
      state: {
        venuesData: wsdcVenue
      }
    };
    this.router.navigate(['venues-map'], navigationExtras);
  }
}

\end{lstlisting} 

\begin{lstlisting}[language=html, label={lst:venues.page.html}, caption=venues.page.html]
<ion-header>
  <ion-toolbar>
    <ion-buttons slot="start">
      <ion-menu-button></ion-menu-button>
    </ion-buttons>
    <ion-title>Venues</ion-title>
  </ion-toolbar>
</ion-header>

<ion-content>
  <ion-grid>
    <ion-row>
      <ion-list no-lines>
        <ion-button color="white" id="{{wsdcVenue.id}}" *ngFor="let wsdcVenue of venuesData" (click)="itemTapped(wsdcVenue)">
          <ion-icon name={{wsdcVenue.icon}}></ion-icon>
          <span>{{ wsdcVenue.name }}</span>
        </ion-button>
      </ion-list>
    </ion-row>
  </ion-grid>
</ion-content>

\end{lstlisting}

\begin{lstlisting}[language=JavaScript, label={lst:venues-map.page.ts}, caption=venues-map.page.ts, escapeinside={(*}{*)}]
import { Component, OnInit} from '@angular/core';
import { ActivatedRoute, Router } from '@angular/router';
import { Storage } from '@ionic/storage';
import { CapacitorGoogleMaps } from "@capacitor-community/google-maps";
import { Geolocation } from '@capacitor/geolocation';

@Component({
  selector: 'app-venues-map',
  templateUrl: './venues-map.page.html',
  styleUrls: ['./venues-map.page.scss'],
})
export class VenuesMapPage{
  venuesPage: any;
  venuesMapsDetail: any;
  userCoordinatesLat: any;
  userCoordinatesLng: any;
  mapid: any;
  userDistanceTo: string[];
  itemCounter: number;
  items: Array<{id: string, idcolor:number }>
  pageTitleColors: Array<string> = ["#ec1c24", "#c49a6c", "#d189bb", "#4d113f"];

  constructor(private activatedRoute: ActivatedRoute, private router: Router,private storage: Storage) {

    //Get data from venues page.
    this.items = [];
    this.activatedRoute.queryParams.subscribe(params => {
      if (this.router.getCurrentNavigation().extras.state) {
        this.venuesPage = this.router.getCurrentNavigation().extras.state.venuesData;
      }
      
      this.items.push({
        id: this.venuesPage.id,
        idcolor: this.venuesPage.colorIdx
      });
      
      document.getElementById("pagetitle").style.color = this.pageTitleColors[this.items[0].idcolor-1];
    });

    // Select data from storage
    this.storage.get('wsdcDataStorage').then((data) => {
      this.venuesMapsDetail = data.venues;
      this.venuesMapsDetail = this.venuesMapsDetail.filter(d=> d.id==this.items[0].id);
    })

    //save user lat & lng location
    const printCurrentPosition = async () => {
      const coordinates = await Geolocation.getCurrentPosition();
      this.userCoordinatesLat = coordinates.coords.latitude;
      this.userCoordinatesLng = coordinates.coords.longitude;
    };
    
    printCurrentPosition();
    Geolocation.requestPermissions();
  }

  ionViewDidEnter() {
    // create map and add marker
    const initializeMap = async () => {
      this.itemCounter = 0;
      await CapacitorGoogleMaps.initialize({
        key: "YOUR_IOS_MAPS_API_KEY",
        devicePixelRatio: window.devicePixelRatio,
      });
      const element = document.getElementById("mapContainer");
      const boundingRect = element.getBoundingClientRect();
      try {
        const result = await CapacitorGoogleMaps.createMap({
          boundingRect: {
            width: Math.round(boundingRect.width),
            height: Math.round(boundingRect.height),
            x: Math.round(boundingRect.x),
            y: Math.round(boundingRect.y),
          },
          cameraPosition:{
            target:{ //Kuta, Bali -8.722396, 115.17671
              latitude:-8.722396, 
              longitude: 	115.17671
            },
            zoom:11
          },
          preferences:{
            controls:{
              isCompassButtonEnabled:true,
              isMyLocationButtonEnabled:true,
              isZoomButtonsEnabled:true
            },
            appearance:{
              isMyLocationDotShown:true
            }
          },
        });
        
        element.style.background = "";
        element.setAttribute("data-maps-id", result.googleMap.mapId);
        this.mapid = result.googleMap.mapId;
        console.log(this.venuesMapsDetail);
        
        //add marker to each location
        for(let venue of this.venuesMapsDetail){
          for(let venuesMarker of venue.geojson.features){
            this.itemCounter +=1;
            const koor = venuesMarker.geometry.coordinates;  
            CapacitorGoogleMaps.addMarker({
              mapId:result.googleMap.mapId,
              position:{
                latitude: koor[1],
                longitude: koor[0],
              },
              preferences:{
                title: venuesMarker.properties.Name
              },
            });     
          }
        }
      } catch (e) {
        alert("Map failed to load");
      }
  
      // Insert distance to each location
      this.userDistanceTo = new Array(this.itemCounter);
      let counter = 0;
      for(let venue of this.venuesMapsDetail){
        for(let venuesMarker of venue.geojson.features){
          const koor = venuesMarker.geometry.coordinates;  
          this.userDistanceTo[counter] = this.computeDistance(this.userCoordinatesLat,koor[1],this.userCoordinatesLng,koor[0]);
          counter+=1;
        }
      }
    };

    (function () {
      initializeMap();
    })();
  }

  backToVenue() {
    this.router.navigate(['venues']);
  }

  featTapped(venuesCoordinates){
    const coordinates = venuesCoordinates;
    CapacitorGoogleMaps.moveCamera({
      mapId:this.mapid,
      cameraPosition:{
        target:{
          latitude: coordinates[1],
          longitude: coordinates[0],
        },
        zoom:15
      },
      duration:800
    });     
  } 

  computeDistance(lat1, lat2, lon1, lon2){
    
    const R = 6371e3; // metres
    const phi1 = lat1 * Math.PI/180; // phi, lambda in radians
    const phi2 = lat2 * Math.PI/180;
    const deltaPhi = (lat2-lat1) * Math.PI/180;
    const deltaLambda = (lon2-lon1) * Math.PI/180;
    
    const a = Math.sin(deltaPhi/2) * Math.sin(deltaPhi/2) +
              Math.cos(phi1) * Math.cos(phi2) *
              Math.sin(deltaLambda/2) * Math.sin(deltaLambda/2);
    const c = 2 * Math.atan2(Math.sqrt(a), Math.sqrt(1-a));
    
    const d = R * c; // in metres

    
    if(d < 1000){
      return Math.floor(d) + " m";
    }else if (d < 100000){
      return Math.floor(d/1000) + " km";
    }else{
      return ">99 km"
    }
  }
}

\end{lstlisting} 

\begin{lstlisting}[language=html, label={lst:venues-map.page.html}, caption=venues-map.page.html]
<ion-header>
  <ion-toolbar>
    <ion-buttons slot="start" (click)="backToVenue()">
      <ion-icon name="arrow-back-outline"></ion-icon>
    </ion-buttons>
    <ion-title>Venues</ion-title>
  </ion-toolbar>
</ion-header>

<div id="container">
  <div id="mapContainer"></div>
</div>

<!-- //deprecated scroll in ionic 5 -->
<ion-content scrollX="true"> 
  <ion-label>
    <h3 #pagetitle id="pagetitle">{{venuesPage.name}}</h3>
  </ion-label>

  <ion-list>
    <ion-item *ngFor="let venue of venuesMapsDetail; let i=index">
      <div>
        <div class="venuesDescContainer" *ngFor="let properties of venue.geojson.features; let j = index">
          <div class="venueDesc" (click)="featTapped(properties.geometry.coordinates)">
            <h2>{{properties.properties.Name}}</h2>
            <p>{{properties.properties.Description}}</p>
          </div>
          <div class="venuesDist" #distance>
            <p>{{userDistanceTo[j]}}</p>
          </div>
        </div>
      </div>
    </ion-item>
  </ion-list>
</ion-content>


\end{lstlisting}