%_____________________________________________________________________________
%=============================================================================
% data.tex v11 (24-07-2020) dibuat oleh Lionov - Informatika FTIS UNPAR
%
% Perubahan pada versi 11 (24-07-2020)
%	- Penambahan enumitem dan nosep untuk semua list, untuk menghemat kertas
%   - Bagian V: penambahan opsi Daftar Kode Program dan Daftar Notasi
%   - Bagian XIV: menjadi Bagian XVI
%   - Bagian XV: menjadi Bagian XVII
%   - Bagian XIV yang baru: untuk pilihan jenis tanda tangan mahasiswa
%   - Bagian XV yang baru: untuk pilihan munculnya tanda tangan digital untuk
%     dosen/pejabat
%_____________________________________________________________________________
%=============================================================================

%=============================================================================
% 								PETUNJUK
%=============================================================================
% Ini adalah file data (data.tex)
% Masukkan ke dalam file ini, data-data yang diperlukan oleh template ini
% Cara memasukkan data dijelaskan di setiap bagian
% Data yang WAJIB dan HARUS diisi dengan baik dan benar adalah SELURUHNYA !!
% Hilangkan tanda << dan >> jika anda menemukannya
%=============================================================================

%_____________________________________________________________________________
%=============================================================================
% 								BAGIAN 0
%=============================================================================
% Entri untuk memperbaiki posisi "DAFTAR ISI" jika tidak berada di bagian 
% tengah halaman. Sayangnya setiap sistem menghasilkan posisi yang berbeda.
% Isilah dengan 0 atau 1 (e.g. \daftarIsiError{1}). 
% Pemilihan 0 atau 1 silahkan disesuaikan dengan hasil PDF yang dihasilkan.
%=============================================================================
\daftarIsiError{0}   
%\daftarIsiError{1}   
%=============================================================================

%_____________________________________________________________________________
%=============================================================================
% 								BAGIAN I
%=============================================================================
% Tambahkan package2 lain yang anda butuhkan di sini
%=============================================================================
\usepackage{booktabs} 
\usepackage{longtable}
\usepackage{amssymb}
\usepackage{todo}
\usepackage{verbatim} 		%multiline comment
\usepackage{pgfplots}
\usepackage{enumitem}
\usepackage{caption}
\usepackage{subcaption}
\usepackage{hyperref}
\usepackage{listings} %code highlighter
\usepackage{color} %use color
\usepackage{ upgreek }
\usepackage{multirow}
%\overfullrule=3mm % memperlihatkan overfull 
%=============================================================================

%_____________________________________________________________________________
%=============================================================================
% 								BAGIAN II
%=============================================================================
% Mode dokumen: menentukan halaman depan dari dokumen, apakah harus mengandung 
% prakata/pernyataan/abstrak dll (termasuk daftar gambar/tabel/isi) ?
% - final 		: hanya untuk buku skripsi, dicetak lengkap: judul ina/eng, 
%   			  pengesahan, pernyataan, abstrak ina/eng, untuk, kata 
%				  pengantar, daftar isi (daftar tabel dan gambar tetap 
%				  opsional dan dapat diatur), seluruh bab dan lampiran.
%				  Otomatis tidak ada nomor baris dan singlespacing
% - sidangakhir	: buku sidang akhir = buku final - (pengesahan + pernyataan +
%   			  untuk + kata pengantar)
%				  Otomatis ada nomor baris dan onehalfspacing 
% - sidang 		: untuk sidang 1, buku sidang = buku sidang akhir - (judul 
%				  eng + abstrak ina/eng)
%				  Otomatis ada nomor baris dan onehalfspacing
% - bimbingan	: untuk keperluan bimbingan, hanya terdapat bab dan lampiran
%   			  saja, bab dan lampiran yang hendak dicetak dapat ditentukan 
%				  sendiri (nomor baris dan spacing dapat diatur sendiri)
% Mode default adalah 'template' yang menghasilkan isian berwarna merah, 
% aktifkan salah satu mode di bawah ini :
%=============================================================================
%\mode{bimbingan} 		% untuk keperluan bimbingan
%\mode{sidang} 			% untuk sidang 1
%\mode{sidangakhir} 	% untuk sidang 2 / sidang pada Skripsi 2(IF)
%\mode{final} 			% untuk mencetak buku skripsi 
%=============================================================================
\mode{sidangakhir}

%_____________________________________________________________________________
%=============================================================================
% 								BAGIAN III
%=============================================================================
% Line numbering: penomoran setiap baris, nomor baris otomatis di-reset ke 1
% setiap berganti halaman.
% Sudah dikonfigurasi otomatis untuk mode final (tidak ada), mode sidang (ada)
% dan mode sidangakhir (ada).
% Untuk mode bimbingan, defaultnya ada (\linenumber{yes}), jika ingin 
% dihilangkan, isi dengan "no" (i.e.: \linenumber{no})
% Catatan:
% - jika nomor baris tidak kembali ke 1 di halaman berikutnya, compile kembali
%   dokumen latex anda
% - bagian ini hanya bisa diatur di mode bimbingan
%=============================================================================
%\linenumber{no} 
\linenumber{yes}
%=============================================================================

%_____________________________________________________________________________
%=============================================================================
% 								BAGIAN IV
%=============================================================================
% Linespacing: jarak antara baris 
% - single	: otomatis jika ingin mencetak buku skripsi, opsi yang 
%			     disediakan untuk bimbingan, jika pembimbing tidak keberatan 
%			     (untuk menghemat kertas)
% - onehalf	: otomatis jika ingin mencetak dokumen untuk sidang
% - double 	: jarak yang lebih lebar lagi, jika pembimbing berniat memberi 
%             catatan yg banyak di antara baris (dianjurkan untuk bimbingan)
% Catatan: bagian ini hanya bisa diatur di mode bimbingan
%=============================================================================
\linespacing{single}
%\linespacing{onehalf}
%\linespacing{double}
%=============================================================================

%_____________________________________________________________________________
%=============================================================================
% 								BAGIAN V
%=============================================================================
% Tidak semua skripsi memuat gambar, tabel, kode program, dan/atau notasi. 
% Untuk skripsi yang tidak memuat hal-hal tersebut, maka tidak diperlukan 
% Daftar Gambar, Daftar Tabel, Daftar Kode Program, dan/atau Daftar Notasi. 
% Sayangnya hal tsb sulit dilakukan secara manual karena membutuhkan 
% kedisiplinan pengguna template.  
% Jika tidak ingin menampilkan satu/lebih daftar-daftar tersebut (misalnya 
% untuk bimbingan), isi dengan "no" (e.g. \gambar{no})
%=============================================================================
\gambar{yes}
%\gambar{no}
%\tabel{yes}
\tabel{no} 
%\kode{yes}
\kode{no} 
%\notasi{yes}
\notasi{no}
%=============================================================================

%_____________________________________________________________________________
%=============================================================================
% 								BAGIAN VI
%=============================================================================
% Pada mode final, sidang da sidangkahir, seluruh bab yang ada di folder "Bab"
% dengan nama file bab1.tex, bab2.tex s.d. bab9.tex akan dicetak terurut, 
% apapun isi dari perintah \bab.
% Pada mode bimbingan, jika ingin:
% - mencetak seluruh bab, isi dengan 'all' (i.e. \bab{all})
% - mencetak beberapa bab saja, isi dengan angka, pisahkan dengan ',' 
%   dan bab akan dicetak terurut sesuai urutan penulisan (e.g. \bab{1,3,2}). 
% Catatan: Jika ingin menambahkan bab ke-3 s.d. ke-9, tambahkan file 
% bab3.tex, bab4.tex, dst di folder "Bab". Untuk bab ke-10 dan 
% seterusnya, harus dilakukan secara manual dengan mengubah file skripsi.tex
% Catatan: bagian ini hanya bisa diatur di mode bimbingan
%=============================================================================
\bab{all}
%=============================================================================

%_____________________________________________________________________________
%=============================================================================
% 								BAGIAN VII
%=============================================================================
% Pada mode final, sidang dan sidangkhir, seluruh lampiran yang ada di folder 
% "Lampiran" dengan nama file lampA.tex, lampB.tex s.d. lampJ.tex akan dicetak 
% terurut, apapun isi dari perintah \lampiran.
% Pada mode bimbingan, jika ingin:
% - mencetak seluruh lampiran, isi dengan 'all' (i.e. \lampiran{all})
% - mencetak beberapa lampiran saja, isi dengan huruf, pisahkan dengan ',' 
%   dan lampiran akan dicetak terurut sesuai urutan (e.g. \lampiran{A,E,C}). 
% - tidak mencetak lampiran apapun, isi dengan "none" (i.e. \lampiran{none})
% Catatan: Jika ingin menambahkan lampiran ke-C s.d. ke-I, tambahkan file 
% lampC.tex, lampD.tex, dst di folder Lampiran. Untuk lampiran ke-J dan 
% seterusnya, harus dilakukan secara manual dengan mengubah file skripsi.tex
% Catatan: bagian ini hanya bisa diatur di mode bimbingan
%=============================================================================
\lampiran{all}
%=============================================================================

%_____________________________________________________________________________
%=============================================================================
% 								BAGIAN VIII
%=============================================================================
% Data diri dan skripsi/tugas akhir
% - namanpm		: Nama dan NPM anda, penggunaan huruf besar untuk nama harus 
%				  benar dan gunakan 10 digit npm UNPAR, PASTIKAN BAHWA 
%				  BENAR !!! (e.g. \namanpm{Jane Doe}{1992710001}
% - judul 		: Dalam bahasa Indonesia, perhatikan penggunaan huruf besar, 
%				  judul tidak menggunakan huruf besar seluruhnya !!! 
% - tanggal 	: isi dengan {tangga}{bulan}{tahun} dalam angka numerik, 
%				  jangan menuliskan kata (e.g. AGUSTUS) dalam isian bulan.
%			  	  Tanggal ini adalah tanggal dimana anda akan melaksanakan 
%				  sidang ujian akhir skripsi/tugas akhir
% - pembimbing	: pembimbing anda, lihat daftar dosen di file dosen.tex
%				  jika pembimbing hanya 1, kosongkan parameter kedua 
%				  (e.g. \pembimbing{\JND}{} ), \JND adalah kode dosen
% - penguji 	: para penguji anda, lihat daftar dosen di file dosen.tex
%				  (e.g. \penguji{\JHD}{\JCD} )
% !!Lihat singkatan pembimbing dan penguji anda di file dosen.tex!!
% Petunjuk: hilangkan tanda << & >>, dan isi sesuai dengan data anda
%=============================================================================
\namanpm{Rajasa Cikal Maulana Solihin}{2017730084}
\tanggal{14}{6}{2022}         %isi bulan dengan angka
\pembimbing{\PAN}{}    
\penguji{\RCP}{} 
%=============================================================================

%_____________________________________________________________________________
%=============================================================================
% 								BAGIAN IX
%=============================================================================
% Judul dan title : judul bhs indonesia dan inggris
% - judulINA: judul dalam bahasa indonesia
% - judulENG: title in english
% Petunjuk: 
% - hilangkan tanda << & >>, dan isi sesuai dengan data anda
% - langsung mulai setelah '{' awal, jangan mulai menulis di baris bawahnya
% - gunakan \texorpdfstring{\\}{} untuk pindah ke baris baru
% - judul TIDAK ditulis dengan menggunakan huruf besar seluruhnya !!
%=============================================================================
\judulINA{Pembuatan Ulang Aplikasi WSDC  (\textit{World School Debating Championship)} 2017 Bali Dengan Ionic 6}
\judulENG{Re-creation of WSDC (World School Debating Championship) 2017 Bali App with Ionic 6}
%_____________________________________________________________________________
%=============================================================================
% 								BAGIAN X
%=============================================================================
% Abstrak dan abstract : abstrak bhs indonesia dan inggris
% - abstrakINA: abstrak bahasa indonesia
% - abstrakENG: abstract in english 
% Petunjuk: 
% - hilangkan tanda << & >>, dan isi sesuai dengan data anda
% - langsung mulai setelah '{' awal, jangan mulai menulis di baris bawahnya
%=============================================================================
\abstrakINA{\textit{World Schools Debating Championships} (WSDC) merupakan sebuah turnamen debat Bahasa Inggris tahunan untuk tim-tim tingkat sekolah menengah yang mewakili berbagai negara. Pada tahun 2017, WSDC diselenggarakan di Bali, Indonesia. Untuk menunjang acara tersebut, diciptakan sebuah aplikasi WSDC 2017 Bali yang memiliki beberapa fitur seperti melihat jadwal acara, melihat pengumuman acara, melihat lokasi dan peta lokasi acara, dan fitur notifikasi. Aplikasi tersebut dibangun menggunakan Ionic Framework versi 3 dengan Cordova dan diimplementasikan menggunakan kerangka JavaScript Angular serta dapat digunakan pada perangkat mobile berbasis Android dan iOS. Karena pada saat skripsi ini dibuat, Ionic Framework versi 3 sudah tidak lagi dikembangkan, dan aplikasi WSDC 2017 Bali sudah diturunkan dari App Store pada perangkat iOS karena tidak mendapat pembaruan sejak lama, maka dari itu dibuatlah aplikasi pembaruan aplikasi WSDC 2017 Bali menggunakan Ionic Framework versi 6.

Aplikasi WSDC 2017 Bali dengan Ionic Framework 6 diimplementasikan dengan menggunakan kerangka JavaScript Angular. Lalu untuk menggunakan fitur \textit{native} dari perangkat menggunakan Capacitor. Capacitor memiliki \textit{plugins} untuk mengakses \textit{native} API pada pada perangkat mobile. Beberapa \textit{plugins} yang digunakan pada aplikasi ini yaitu: Geolocation API, Google Maps API, Splash Screen, dan Browser API. Dengan digunakannya Capacitor maka aplikasi WSDC 2017 Bali yang ditulis menggunakan bahasa pemrograman web seperti HTML, CSS, dan JavaScript, dapat dijalankan pada perangkat \textit{mobile} dengan sistem operasi Android dan iOS. Namun pada skripsi ini hanya akan diuji pada perangkat Android.

Pengujian aplikasi WSDC 2017 Bali dilakukan terhadap beberapa responden dengan cara mengunduh, menginstal, dan membandingkan aplikasi WSDC 2017 Bali terdahulu dan terbaru. Dari pengujian tersebut didapatkan hasil yaitu aplikasi dapat berjalan dengan lancar pada perangkat Android mulai dari versi 5.1 sampai dengan versi 11. Dengan begitu, aplikasi WSDC 2017 Bali dengan Ionic Framework versi 6 telah berhasil dibangun.}

\abstrakENG{The World Schools Debating Championships (WSDC) is an annual English language debate tournament for high school-level teams representing various countries. In 2017, WSDC was held in Bali, Indonesia. To support the event, a WSDC 2017 Bali application was created which has several features such as viewing the event schedule, displaying announcements, viewing the location and map of the event location, and notification features. The application was built using the Ionic Framework version 3 with Cordova and implemented using the Angular JavaScript framework. At the time this thesis was written, the Ionic Framework version 3 was no longer developed, and the Ionic Framework had reached version 6. Therefore, an update for the WSDC 2017 Bali application will be made using the Ionic Framework version 6.

The WSDC 2017 Bali application with Ionic Framework 6 is implemented using the Angular JavaScript framework. Then to use the native features of devices that use Capacitor. Capacitor has a plugin to access the full native API. Some of the plugins used in this application are: Geolocation API, Google Maps API, Splash Screen, and Browser API. With the Capacitor, the WSDC 2017 Bali application, which was originally created using a web programming language, can be implemented to run on Android and iOS operating systems. However, in this thesis, it will only be tested on Android devices.

Testing of the WSDC 2017 Bali application was carried out on several respondents by downloading, installing, and comparing the previous and latest WSDC 2017 Bali applications. From these tests, the results are that the application can run well on Android devices starting from version 5.1 to version 11, along with the existing features. That way, the recreation of the WSDC 2017 Bali application with the Ionic Framework version 6 has been successfully built.} 
%=============================================================================

%_____________________________________________________________________________
%=============================================================================
% 								BAGIAN XI
%=============================================================================
% Kata-kata kunci dan keywords : diletakkan di bawah abstrak (ina dan eng)
% - kunciINA: kata-kata kunci dalam bahasa indonesia
% - kunciENG: keywords in english
% Petunjuk: hilangkan tanda << & >>, dan isi sesuai dengan data anda.
%=============================================================================
\kunciINA{WSDC, Ionic Framework, UI Component, Capacitor, Cordova, Angular}
\kunciENG{WSDC, Ionic Framework, UI Component, Capacitor, Cordova, Angular}
%=============================================================================

%_____________________________________________________________________________
%=============================================================================
% 								BAGIAN XII
%=============================================================================
% Persembahan : kepada siapa anda mempersembahkan skripsi ini ...
% Petunjuk: hilangkan tanda << & >>, dan isi sesuai dengan data anda.
%=============================================================================
\untuk{Untuk Mimih dan Pipih, keluarga, dan diri saya sendiri.}
%=============================================================================

%_____________________________________________________________________________
%=============================================================================
% 								BAGIAN XIII
%=============================================================================
% Kata Pengantar: tempat anda menuliskan kata pengantar dan ucapan terima 
% kasih kepada yang telah membantu anda bla bla bla ....  
% Petunjuk: hilangkan tanda << & >>, dan isi sesuai dengan data anda.
%=============================================================================
\prakata{Puji dan syukur saya panjatkan kepada Allah SWT. atas ridanya sehingga penulis dapat menyelesaikan penyusunan skripsi yang berjudul ``Pembuatan Ulang Aplikasi WSDC 2017 Bali Dengan Ionic 5'' dengan baik. Skripsi ini disusun untuk memenuhi syarat kelulusan jurusan Teknik Informatik di Fakultas Teknologi Informasi dan Sains, Universitas Katolik Parahyangan. Penulis juga ingin berterimakasih terhadap beberapa pihak yang telah membantu serta memberi dukungan terhadap penulis dalam menyusun skripsi ini, yaitu: 

\begin{enumerate}
	\item Mimih dan pipih, serta keluarga yang selalu memberi motivasi, dukungan, serta doa kepada penulis selama ini.
	\item Bapak Pascal Alfadian, Nugroho, M.Comp. sebagai dosen pembimbing yang telah memberikan dukungan serta arahan terhadap penulis sehingga dapat menyelesaikan skripsi ini dengan baik.
	\item Dosen Penguji sebagai dosen penguji yang telah memberikan kritik dan saran kepada penulis.
	\item Segenap dosen Teknik Informatika UNPAR yang telah memberikan ilmu dan pelajaran bagi penulis selama berada di UNPAR.
	\item Orang terkasih, Astry Destiana Rhosyta, yang selalu mendukung dan menemani penulis selama pengerjaan skripsi ini.
	\item Teman-teman Teknik Informatika UNPAR yang telah berbagi ilmu dan memberikan dukungan kepada penulis selama ini.
	\item Pihak-pihak lain yang telah membantu dalam pengujian aplikasi WSDC 2017 Bali.
\end{enumerate}

Penulis berharap semoga skripsi ini dapat bermanfaat bagi pembaca dan pihak-pihak yang membutuhkan bagi pengembangan lebih lanjut terkait skripsi ini.
\ldots} 
%=============================================================================

%_____________________________________________________________________________
%=============================================================================
% 								BAGIAN XIV
%=============================================================================
% Jenis tandatangan di lembar pernyataan mahasiswa tentang plagiarisme.
% Ada 4 pilihan:
%   - digital   : diisi menggunakan digital signature (menggunakan pengolah
%                 pdf seperti Adobe Acrobat Reader DC).
%   - gambar    : diisi dengan gambar tandatangan mahasiswa (file tandatangan
%                 bertipe pdf/png/jpg). Dianjukan menggunakan warna biru.
%                 Letakkan gambar di folder "Gambar" dengan nama ttd.jpg/
%                 ttd.png/ttd.pdf (tergantung jenis file. Hapus file ttd.jpg
%                 yang digunakan sebagai contoh
%   - materai   : khusus bagi yang ingin mencetak buku dan menandatangani di 
%                 atas materai. Sama dengan pilihan ``digital'' dan dicetak.
%   - kosong    : tempat kosong ini bisa diisi dengan tanda tangan yang
%                 digambar langsung di atas pdf (fill&sign via acrobat, tanda
%                 tangan dapat dibuat dengan mouse atau stylus)
%=============================================================================
%\ttd{digital}
%\ttd{gambar}
%\ttd{materai}
%\ttd{kosong}
%=============================================================================
\ttd{gambar}

%_____________________________________________________________________________
%=============================================================================
% 								BAGIAN XV
%=============================================================================
% Pilihan tanda tangan digital untuk dosen/pejabat:
%   - no    : pdf TIDAK dapat ditandatangani secara digital, mengakomodasi 
%             yang akan menandatangani via ``menulis'' di file pdf
%   - yes   : pdf dapat ditandatangani secara digital
% 
% PERHATIAN: perubahan ini harus ditanyakan ke kaprodi/dosen koordinator, 
% apakah harus mengisi ``no" atau ``yes". Default = no 
% Untuk mahasiswa Informatika = yes
%=============================================================================
%\ttddosen{yes}
%=============================================================================
\ttddosen{no}

%_____________________________________________________________________________
%=============================================================================
% 								BAGIAN XVI
%=============================================================================
% Tambahkan hyphen (pemenggalan kata) yang anda butuhkan di sini 
%=============================================================================
\hyphenation{ma-te-ma-ti-ka}
\hyphenation{fi-si-ka}
\hyphenation{tek-nik}
\hyphenation{in-for-ma-ti-ka}
%=============================================================================

%_____________________________________________________________________________
%=============================================================================
% 								BAGIAN XVII
%=============================================================================
% Tambahkan perintah yang anda buat sendiri di sini 
%=============================================================================
\renewcommand{\vtemplateauthor}{lionov}
\pgfplotsset{compat=newest}
\setlist{nosep}
\definecolor{mygreen}{rgb}{0,0.6,0}
\definecolor{mygray}{rgb}{0.5,0.5,0.5}
\definecolor{mymauve}{rgb}{0.58,0,0.82}
\lstset{ %
backgroundcolor=\color{white}, % choose the background color; you must add \usepackage{color} or \usepackage{xcolor}
basicstyle=\footnotesize, % the size of the fonts that are used for the code
breakatwhitespace=false, % sets if automatic breaks should only happen at whitespace
breaklines=true, % sets automatic line breaking
captionpos=b, % sets the caption-position to bottom
commentstyle=\color{mygreen}, % comment style
deletekeywords={...}, % if you want to delete keywords from the given language
escapeinside={\%*}{*)}, % if you want to add LaTeX within your code
extendedchars=true, % lets you use non-ASCII characters; for 8-bits encodings only, does not work with UTF-8
frame=single, % adds a frame around the code
keepspaces=true, % keeps spaces in text, useful for keeping indentation of code (possibly needs columns=flexible)
keywordstyle=\color{blue}, % keyword style
% language=Octave, % the language of the code
morekeywords={*,...}, % if you want to add more keywords to the set
numbers=left, % where to put the line-numbers; possible values are (none, left, right)
numbersep=5pt, % how far the line-numbers are from the code
numberstyle=\tiny\color{mygray}, % the style that is used for the line-numbers
rulecolor=\color{black}, % if not set, the frame-color may be changed on line-breaks within not-black text (e.g. comments (green here))
showspaces=false, % show spaces everywhere adding particular underscores; it overrides 'showstringspaces'
showstringspaces=false, % underline spaces within strings only
showtabs=false, % show tabs within strings adding particular underscores
stepnumber=1, % the step between two line-numbers. If it's 1, each line will be numbered
stringstyle=\color{mymauve}, % string literal style
tabsize=2, % sets default tabsize to 2 spaces
title=\lstname % show the filename of files included with \lstinputlisting; also try caption instead of title
}
%END of listing package%
 
\definecolor{darkgray}{rgb}{.4,.4,.4}
\definecolor{purple}{rgb}{0.65, 0.12, 0.82}
 
%define Javascript language
\lstdefinelanguage{JavaScript}{
keywords={typeof, new, true, false, catch, function, return, null, catch, switch, var, if, in, while, do, else, case, break},
keywordstyle=\color{blue}\bfseries,
ndkeywords={class, export, boolean, throw, implements, import, this},
ndkeywordstyle=\color{darkgray}\bfseries,
identifierstyle=\color{black},
sensitive=false,
comment=[l]{//},
morecomment=[s]{/*}{*/},
commentstyle=\color{purple}\ttfamily,
stringstyle=\color{red}\ttfamily,
morestring=[b]',
morestring=[b]"
}
 
\lstset{
language=JavaScript,
extendedchars=true,
basicstyle=\footnotesize\ttfamily,
showstringspaces=false,
showspaces=false,
numbers=left,
numberstyle=\footnotesize,
numbersep=9pt,
tabsize=2,
breaklines=true,
showtabs=false,
captionpos=b
}
%=============================================================================