%versi 2 (8-10-2016) 
\chapter{Pendahuluan}
\label{chap:intro}
   
\section{Latar Belakang}
\label{sec:label}

\textit{World Schools Debating Championships} (WSDC) merupakan sebuah turnamen debat Bahasa Inggris tahunan untuk tim-tim tingkat sekolah menengah yang mewakili berbagai negara~\cite{wsdc}. Pada awalnya, kompetisi universitas dunia diselenggarakan di Sydney pada bulan Juli 1988. Anggota Federasi Debat Australia menyadari bahwa tidak ada acara serupa untuk siswa sekolah menengah. Namun kejuaraan universitas dunia ini menunjukkan potensi yang sangat besar untuk kompetisi debat internasional yang melibatkan siswa dari seluruh dunia. Pada tahun 1991, kejuaraan diadakan di Edinburgh. Dan sejak saat itu nama World Schools Debating Championships digunakan dan berlangsung hingga saat ini. 

WSDC yang diselenggarakan di Bali, Indonesia pada tahun 2017 memiliki sebuah aplikasi bernama WSDC 2017 Bali yang dikembangkan oleh PT DNArtworks Komunikasi Visual menggunakan \textit{framework} Ionic 3 untuk menunjang acara tersebut. Terdapat beberapa fungsi penting di dalam aplikasi ini, diantaranya adalah jadwal untuk kegiatan peserta, berita tentang acara WSDC yang sedang berlangsung, pemberitahuan mengenai kegiatan acara kepada peserta, informasi lokasi dan peta lokasi kegiatan acara yang sedang berlangsung, dan notifikasi untuk peserta. 

Ionic Framework merupakan sebuah {\it framework open source} lintas platform yang digunakan untuk mengembangkan aplikasi \textit{hybrid} yang bekerja pada berbagai macam platform seluler seperti Android, iOS, dan Windows~\cite{waranashiwar:18:ionic}. Aplikasi \textit{hybrid} merupakan sebuah WebApp yang berjalan di dalam sebuah WebView namun dapat menggunakan fitur-fitur yang disediakan oleh perangkat \textit{mobile}~\cite{yusuf:16:ionic}. Dalam kata lain, aplikasi \textit{hybrid} merupakan sebuah aplikasi \textit{native} yang menampilkan \textit{web app} di dalam \textit{web view}. Perbedaan \textit{web app} dengan aplikasi \textit{native} adalah dalam hal penggunaan perangkat keras pada perangkat mobile, dimana aplikasi \textit{native} dapat dengan bebas mendapatkan akses penuh penggunaan perangkat keras sedangkan \textit{web app} tidak.

Salah satu aplikasi \textit{hybrid} yaitu Apache Cordova yang dikembangkan oleh Adobe digunakan untuk memecahkan masalah dimana pada saat mengembangkan suatu aplikasi seluler, setiap vendor perangkat lunak memilki alat-alat yang unik yang hanya dimiliki oleh vendor tersebut, yaitu \textit{Software Development Kit} (SDK)~\cite{yusuf:16:ionic}. Karena SDK yang digunakan berbeda-beda tergantung kepada jenis sistem operasi perangkat seluler, maka tidak dapat membuat aplikasi untuk platform yang berbeda, namun dengan baris kode yang sama. Apache Cordova sebagai aplikasi \textit{hybrid} dapat membuat aplikasi berbasis web seperti HTML, {\it Cascading Style Sheets} (CSS), dan Javascript, dikemas sebagai aplikasi \textit{native} yang dapat mengakses fitur-fitur perangkat keras dari suatu perangkat.

Ionic Framework mendukung komunikasi dengan menggunakan Native API, yaitu pengembangan aplikasi langsung terintegrasi ke dalam platform~\cite{griffith:17:mobile}. Cordova merupakan Native API yang digunakan untuk menambahkan fungsionalitas ke dalam aplikasi Ionic apapun. Selain Cordova, terdapat Native API lain yang didukung oleh Ionic Framework, yaitu Capacitor. Capacitor merupakan penerus dari Cordova yang menyediakan akses ke perangkat {\it native} dan fitur platform, serta untuk menyediakan satu set API untuk mengembangkan aplikasi seluler secara \textit{hybrid}, {\it Progressive Web Apps} berbasis web, dan aplikasi komputer berbasis Electron~\cite{tor:19:software}. Ionic juga memiliki berbagai macam \textit{front-end library} dan \textit{User Interface}(UI) {\it Components} berupa \textit{tag} khusus yang digunakan untuk menambah fungsionalita aplikasi.

\newpage

Pada Ionic 5 ke atas, terdapat beberapa \textit{framework} Javascript yang dapat diimplementasikan menggunakan \textit{framework} Ionic, yaitu:

\begin{itemize}
	\item Angular \\
	Angular pada awalnya diciptakan oleh karyawan Google, Misko Hevert dan Adam Abrons pada tahun 2008, yang masih bernama AngularJS dan dikembangkan dengan JavaScript~\cite{wohlgethan:18:supporting}. Pada saat AngularJS pertama kali diciptakan, sebagian besar situs web menggunakan aplikasi multi-halaman, yaitu ketika pengguna mengklik tautan, maka browser harus mengambil dokumen HTML yang diminta dari server. Angular tidak mengimplementasi hal tersebut, melainkan menggunakan \textit{Single-page Application} (SPA), yaitu ketika halaman awal dimuat, semua yang dibutuhkan untuk membuat dan menampilkan sebuah halaman diunduh, kemudian ditampilkan kedalam layar. Dengan begitu, \textit{browser} tidak perlu  mengunduh ulang yang dibutuhkan saat menampilkan halaman~\cite{scott:15:spa}.
	
	\item React \\
	React adalah \textit{library} JavaScript {\it open source} untuk membangun antarmuka pengguna, dikelola oleh Facebook, dapat digunakan dalam berbagai skenario termasuk aplikasi iOS dan Android~\cite{wohlgethan:18:supporting}.
	
	\item Vue \\
	 Vue merupakan \textit{framework} progresif untuk membangun antarmuka pengguna untuk web, yang dapat digunakan baik untuk projek kecil dan untuk {\it Single-Page Applications} (SPAs)~\cite{wohlgethan:18:supporting}.
\end{itemize}

Aplikasi WSDC 2017 Bali yang dibangun pada tahun 2017 oleh PT DNArtworks Komunikasi Visual untuk perangkat iOS untuk sistem operasi iOS telah diturunkan dari App Store dikarenakan tidak mengalami pembaruan dalam jangka waktu tertentu. Maka dari itu diperlukan pembaruan pada aplikasi WSDC 2017 Bali. Pembaruan tersebut dilakukan dengan memperbarui versi Ionic Framework yang sebelumnya versi 3, menjadi versi 6. Pada awalnya pembaruan hanya sampai Ionic Framework versi 5, namun karena Ionic Framework versi 6 diluncurkan pada skripsi ini dibuat, maka dari itu pembaruan dilanjutkan ke Ionic Framework versi 6. Pembaruan Ionic Framework diperlukan karena Ionic versi 3 sudah tidak mendapat dukungan lagi dari tim pengembang Ionic Framework. Untuk melakukan pembaruan tersebut, maka pada skripsi ini dibuat sebuah aplikasi WSDC 2017 Bali baru, yang merupakan sebuah pembaruan dari aplikasi WSDC 2017 Bali. Pembaruan dilakukan dengan cara membuat aplikasi WSDC 2017 Bali baru menggunakan \textit{framework} Ionic versi 6 dan Capacitor dengan fitur-fitur dan halaman yang sama seperti aplikasi sebelumnya. Dilakukan juga penyesuaian pembaruan yang terjadi pada saat melakukan pembaruan dari Ionic 3 ke Ionic 6 serta pembaruan pada saat mengganti Cordova dengan Capacitor. 

\section{Rumusan Masalah}
\label{sec:rumusan}
Rumusan masalah yang dibahas pada skripsi ini yaitu:
\begin{enumerate}
	\item Bagaimana melakukan migrasi aplikasi Android WSDC 2017 Bali ke {\it framework} Ionic versi 6?
	\item Bagaimana menjalankan aplikasi WSDC 2017 Bali pada perangkat Android setelah dilakukan migrasi?
\end{enumerate}


\section{Tujuan}
\label{sec:tujuan}
Tujuan yang ingin dicapai dari penulisan skripsi ini yaitu:
\begin{enumerate}
	\item Melakukan migrasi aplikasi Android WSDC 2017 Bali ke {\it framework} Ionic versi 6.
	\item Menjalankan aplikasi WSDC 2017 Bali pada perangkat Android setelah dilakukan migrasi.
\end{enumerate}


\section{Batasan Masalah}
\label{sec:batasan}
Dalam skripsi ini dibuat batasan-batasan masalah dalam pembuatan perangkat lunak.  Batasan-batasan masalah yang ditetapkan adalah sebagai berikut:

\begin{enumerate}
    \item Aplikasi ini tidak akan memiliki fitur notifikasi, dikarenakan sudah tidak terdapat \textit{service} dari Ionic dan tidak dikembangkan lagi dari sisi servernya. 
    
    \item Walaupun Ionic Framework mendukung pengembangan aplikasi untuk sistem operasi iOS, aplikasi hanya akan diuji pada \textit{platform mobile} berbasis android.
    
\end{enumerate}


\section{Metodologi}
\label{sec:metlit}

Langkah-langkah yang dilakukan dalam skripsi ini adalah sebagai berikut:

\begin{enumerate}
	\item Melakukan studi mengenai {\it framework} Ionic versi 3 dan versi 6.
	\item Menganalisis aplikasi WSDC 2017 Bali.
	\item Mempelajari bagaimana cara melakukan migrasi Ionic versi 3 ke versi 6.
	\item Membangun aplikasi WSDC dengan {\it framework} Ionic versi 6. 
	\item Melakukan pengujian dan eksperimen.
	\item Menulis dokumen skripsi.
\end{enumerate}


\section{Sistematika Pembahasan}
\label{sec:sispem}

Sistematika penulisan setiap bab pada skripsi ini adalah sebagai berikut:
\begin{enumerate}
	\item Bab Pendahuluan \\
	Bab 1 berisi latar belakang, rumusan masalah, tujuan, batasan masalah, metodologi, dan sistematika pembahasan yang digunakan untuk menyusun skripsi ini.
	\item Bab Dasar Teori \\
	Bab 2 berisi teori-teori yang digunakan dalam pembuatan skripsi ini. Teori-teori tersebut yaitu WSDC, Angular, Ionic Framework, Capacitor, Cordova, UI Components, dan Migrasi Ionic.
	\item Bab Analisis \\
	Bab 3 berisi analisis yang dilakukan pada skripsi ini, meliputi analisis sistem kini, analisis kebutuhan aplikasi WSDC 2017 Bali yang akan dibangun, serta permasalahan pembangunan sistem usulan.
	\item Bab Perancangan \\
	Bab 4 berisi perancangan aplikasi meliputi perancangan kelas beserta dengan diagram kelas, deskripsi kelas dan fungsinya, serta perancangan struktur HTML.
	\item Bab Implementasi dan Pengujian \\
	Bab 5 berisi implementasi dan pengujian aplikasi meliputi lungkungan implementasi, hasil implementasi, pengujian fungsional, dan pengujian eksperimental.
	\item Bab Kesimpulan dan Saran \\
	Bab 6 berisi kesimpulan dari hasil pembangunan aplikasi ini dan saran untuk pengembangan selanjutnya.
	\end{enumerate}
