%versi 2 (8-10-2016) 
\chapter{Pendahuluan}
\label{chap:intro}
   
\section{Latar Belakang}
\label{sec:label}

\textit{World Schools Debating Championships} (WSDC) merupakan sebuah turnamen debat bahasa inggris tahunan untuk tim-tim tingkat sekolah menengah yang mewakili berbagai negara \footnote{\textit{`WSDC'} https://wsdcdebate.org/history, Diakses pada 8 Juli 2021. \label{ref:wsdc}}. Pada awalnya, kompetisi universitas dunia akan diselenggarakan di Sydney pada bulan juli 1988. Anggota Federasi Debat Australia menyadari bahwa tidak ada acara serupa untuk siswa sekolah menengah. Namun kejuaraan universitas dunia ini menunjukkan potensi yang sangat besar untuk kompetisi debat internasional yang melibatkan siswa dari seluruh dunia. Pada tahun 1991, kejuaraan diadakan di Edinburgh. Dan sejak saat itu nama World Schools Debating Championships digunakan dan berlangsung hingga saat ini. 

Ionic merupakan {\it Software Development Kit} (SDK) {\it open source} yang digunakan untuk pengembangan aplikasi seluler yang bersifat hibrida pada tahun 2013 oleh Max Lynch, Ben Sperry, dan Adam Bradley, dibangun menggunakan AngularJS dan Apache Cordova \footnote{\textit{'Ionic'}, https://en.wikipedia.org/wiki/Ionic\textunderscore(mobileg\textunderscore app framework) Diakses pada 2 Oktober 2021}. Ionic memiliki berbagai macam \textit{front-end library} dan \textit{User Interface}(UI) {\it Components} yang digunakan untuk  perancangan aplikasi menggunakan teknologi web seperti HTML, CSS, dan Javascript. HTML (\textit{Hyper Text Markup Language}) adalah bahasa markup yang mendefinisikan struktur suatu konten aplikasi \footnote{\textit{`HTML'}, https://developer.mozilla.org/en-US/docs/Learn/Getting\textunderscore started\textunderscore with\textunderscore the\textunderscore web/HTML\textunderscore basics Diakses pada 18 Juli 2021 \label{ref:htmldefinition}}. Untuk melengkapi HTML, menggunakan CSS sebagai pelengkap yang mempresentasikan tampilan HTML ke layar atau ke media lain. CSS juga dapat menghemat banyak pekerjaan, dan dapat mengontrol tata letak beberapa halaman sekaligus. Selain itu ada juga Javascript, yaitu bahasa skrip atau pemrograman yang memungkinkan kita untuk mengimplementasikan fitur-fitur yang kompleks pada halaman web, menampilkan pembaruan konten sesuai dengan waktu yang diinginkan secara interaktif, animasi 2 dimensi atau grafik 3 dimensi, video, dan lain-lain. \footnote{\textit{`What is Javascript?'} \\ https://developer.mozilla.org/en-US/docs/Learn/JavaScript/First\textunderscore steps/What\textunderscore is\textunderscore JavaScript, \\ Diakses pada 8 Juli 2021.} Ketiga elemen ini digunakan dalam \textit{framework} Ionic yang akan dikerjakan.

\begin{figure}[H]
    \centering
    \includegraphics[scale=0.12]{Gambar/ionicLogo.png}
    \caption{Logo Ionic \ref{ref:ionicwebsite}}
    \label{fig:ionic-logo}
\end{figure}

\newpage

Terdapat beberapa kerangka Javascript yang dapat diimplementasikan menggunakan \textit{framework} Ionic, seperti Angular, React, dan Vue. Angular merupakan sebuah \textit{development platform}, yang dibangun di atas TypeScript \footnote{\textit{`What is Angular?'}, https://angular.io/guide/what-is-angular Diakses pada 18 Juli 2021}. Angular meliputi kerangka kerja berbasis komponen untuk membangun aplikasi web yang skalabel, kumpulan \textit{library} yang terintegrasi dengan baik yang mencakup berbagai fitur termasuk perutean, manajemen formulir, komunikasi klien-server, dan masih banyak lagi. Angular pun merupakan serangkaian alat pengembang untuk membantu mengembangkan, menguji, dan memperbarui kode yang dibuat. React adalah \textit{library} JavaScript {\it open source} untuk membangun antarmuka pengguna atau komponen UI, dikelola oleh Facebook dan komunitas pengembang individu dan perusahaan, dapat digunakan sebagai dasar dalam pengembangan aplikasi satu halaman atau aplikasi seluler\footnote{\textit{`React (JavaScript Library'} ,https://en.wikipedia.org/wiki/React\textunderscore(JavaScript\textunderscore library) Diakses pada 3 Oktober 2021}. Sedangkan Vue merupakan \textit{framework}  progresif untuk membangun antarmuka pengguna, yang dirancang dari bawah ke atas agar dapat diadopsi secara bertahap \footnote{\textit{`Vue.js'}, https://vuejs.org/v2/guide/ Diakses pada 18 Juli 2021}. Inti dari \textit{library} ini ada pada lapisan tampilan, yang dapat diintegrasikan dengan \textit{library} lain atau proyek yang sudah ada. 

WSDC yang diselenggarakan pada tahun 2017 memiliki sebuah aplikasi bernama WSDC 2017 Bali menggunakan \textit{framework} Ionic 3. Terdapat beberapa fungsi penting di dalam aplikasi ini, diantaranya adalah jadwal untuk kegiatan peserta, berita acara, pemberitahuan mengenai kegiatan acara kepada peserta, lokasi kegiatan acara yang sedang berlangsung, dan notifikasi acara. 

Aplikasi ini dibangun pada tahun 2017 menggunakan Ionic versi 3, dan sudah tidak mendapatkan pembaruan lagi. Tetapi, versi Ionic semakin berkembang, dan saat ini telah mencapai versi 5. Maka dari itu, pada skripsi ini akan dibuat sebuah aplikasi pembaruan dari aplikasi WSDC 2017 Bali yang sudah ada, dengan \textit{framework} Ionic versi 5. \textit{Framework} yang lebih baru memungkinkan perawatan yang lebih efisien, serta dukungan teknologi yang lebih terbarukan.

\newpage
\section{Rumusan Masalah}
\label{sec:rumusan}
Rumusan masalah yang akan dibahas pada skripsi ini adalah sebagai berikut :
\begin{itemize}
	\item Fitur-fitur apa yang akan tersedia di aplikasi WSDC terbaru?
	\item Bagaimana membangun aplikasi {\it android} WSDC menggunakan {\it framework} Ionic versi 5?
	\item Bagaimana melakukan migrasi Ionic versi 3 ke Ionic versi 5?
\end{itemize}


\section{Tujuan}
\label{sec:tujuan}
Tujuan yang ingin dicapai dari penulisan skripsi ini adalah sebagai berikut :
\begin{itemize}
	\item Mendefinisikan fitur-fitur yang akan tersedia di aplikas WSDC terbaru.
	\item Membangun aplikasi {\it android} WSDC menggunakan {\it framework} Ionic versi 5.
	\item Melakukan migrasi Ionic versi 3 ke Ionic versi 5.
\end{itemize}


\section{Batasan Masalah}
\label{sec:batasan}
Dalam skripsi ini dibuat batasan-batasan masalah dalam pembuatan perangkat lunak.  Batasan-batasan masalah yang ditetapkan adalah sebagai berikut:

\begin{enumerate}
    \item Aplikasi ini tidak akan memiliki fitur notifikasi.
    \item Aplikasi hanya akan berjalan pada \textit{platform mobile} berbasis android. 
\end{enumerate}


\section{Metodologi}
\label{sec:metlit}

Langkah-langkah yang dilakukan dalam skripsi ini adalah:

\begin{enumerate}
		\item Melakukan studi mengenai {\it framework} Ionic versi 3 dan versi 5.
		\item Menganalisis aplikasi WSDC 2017 Bali.
		\item Mempelajari bagaimana cara melakukan migrasi Ionic versi 3 ke versi 5.
		\item Mendesain kelas aplikasi.
		\item Membangun aplikasi WSDC dengan {\it framework} Ionic versi 5. 
		\item Melakukan pengujian dan eksperimen.
		\item Menulis dokumen skripsi.
	\end{enumerate}


\section{Sistematika Pembahasan}
\label{sec:sispem}

Sistematika penulisan setiap bab pada skripsi ini adalah sebagai berikut:
\begin{enumerate}
	\item Bab Pendahuluan \\
	Bab 1 berisi latar belakang, rumusan masalah, tujuan, batasan masalah, metodologi, dan sistematika pembahasan yang digunakan untuk menyusun skripsi ini.
	\item Bab Dasar Toeri \\
	Bab 2 berisi teori-teori yang digunakan dalam pembuatan skripsi ini. Teori-teori tersebut yaitu
	\end{enumerate}