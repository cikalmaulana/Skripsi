%versi 2 (8-10-2016) 
\chapter{Pendahuluan}
\label{chap:intro}
   
\section{Latar Belakang}
\label{sec:label}

Dalam pengembangan suatu aplikasi seluler, setiap vendor perangkat lunak memilki alat-alat yang unik yang hanya dimiliki oleh vendor tersebut, yaitu \textit{Software Development Kit} (SDK)~\cite{yusuf:16:ionic}. Setiap sistem operasi memiliki perangkat \textit{native} masing-masing, yang digunakan sebagai alat untuk memungkinkan pengembang dalam pengembangan aplikasi seluler pihak ketiga. Namun karena SDK yang digunakan berbeda-beda tergantung kepada jenis sistem operasi perangkat seluler, maka terdapat permasalahan, yaitu tidak dapat membuat aplikasi untuk platform yang berbeda, namun dengan baris kode yang sama. Untuk mengatasi masalah tersebut, terdapat sebuah teknologi yang dikembangkan oleh Adobe yaitu Apache Cordova. Apache Cordova memungkinkan aplikasi berbasis web seperti HTML, CSS, dan Javascript, dikemas sebagai aplikasi seluler \textit{native}, dan juga menyediakan akses ke fitur perangkat.

Aplikasi yang dibangun menggunakan Cordova disebut sebagai aplikasi \textit{hybrid}, sebuah aplikasi berbasis web yang berjalan di dalam sebuah tempat dari aplikasi \textit{native}~\cite{yusuf:16:ionic}. Aplikasi \textit{hybrid} yang berjalan merupakan aplikasi \textit{native} yang menjalankan aplikasi berbasis web di tampilan web ponsel. Aplikasi tersebut bersikap dan berperilaku normal seperti aplikasi pada umumnya, dan juga memiliki keseluruhan akses terhadap perangkat. 

Salah satu kerangka kerja yang dapat digunakan untuk mengembangkan aplikasi \textit{hybrid} adalah Ionic Framework. Ionic Framework merupakan sebuah kerangka kerja {\it open source} lintas platform yang memungkinkan untuk mengembangkan aplikasi \textit{hybrid} yang bekerja pada berbagai macam platform seluler seperti {\it android}, iOS, dan Windows~\cite{waranashiwar:18:ionic}. Ionic memiliki berbagai macam \textit{front-end library} dan \textit{User Interface}(UI) {\it Components} yang digunakan untuk  perancangan aplikasi menggunakan teknologi web seperti HTML, {\it Cascading Style Sheets} (CSS), dan Javascript. Dengan Ionic Framework, memungkinkan sebuah aplikasi yang dapat berjalan menggunakan fitur \textit{native} dari sebuah perangkat.

Pada Ionic 5, terdapat beberapa kerangka Javascript yang dapat diimplementasikan menggunakan \textit{framework} Ionic, seperti Angular, React, dan Vue. Angular pada awalnya diciptakan oleh karyawan Google, Misko Hevert dan Adam Abrons pada tahun 2008, yang masih bernama AngularJS dan dikembangkan dalam JavaScript~\cite{wohlgethan:18:supporting}. Pada saat itu sebagian besar situs web menggunakan aplikasi multi-halaman, yaitu ketika pengguna mengklik tautan, maka browser harus mengambil dokumen HTML yang diminta dari server. Namun dengan Angular, digunakannya \textit{Single-page Application} (SPA), yaitu ketika halaman awal dimuat, semua yang dibutuhkan untuk membuat dan menampilkan sebuah halaman diunduh, kemudian ditampilkan kedalam layar. Dengan begitu, browser tidak perlu melakukan \textit{refresh} terhadap halaman tersebut~\cite{scott:15:spa}. React adalah \textit{library} JavaScript {\it open source} untuk membangun antarmuka pengguna, dikelola oleh Facebook, dapat digunakan dalam berbagai skenario termasuk aplikasi iOS dan Android~\cite{wohlgethan:18:supporting}. Sedangkan Vue merupakan \textit{framework}  progresif untuk membangun antarmuka pengguna untuk web, yang dapat digunakan baik untuk projek kecil dan untuk {\it Single-Page Applications} (SPAs)~\cite{wohlgethan:18:supporting}.

\textit{World Schools Debating Championships} (WSDC) merupakan sebuah turnamen debat bahasa inggris tahunan untuk tim-tim tingkat sekolah menengah yang mewakili berbagai negara~\cite{wsdc}. Pada awalnya, kompetisi universitas dunia akan diselenggarakan di Sydney pada bulan Juli 1988. Anggota Federasi Debat Australia menyadari bahwa tidak ada acara serupa untuk siswa sekolah menengah. Namun kejuaraan universitas dunia ini menunjukkan potensi yang sangat besar untuk kompetisi debat internasional yang melibatkan siswa dari seluruh dunia. Pada tahun 1991, kejuaraan diadakan di Edinburgh. Dan sejak saat itu nama World Schools Debating Championships digunakan dan berlangsung hingga saat ini. 

WSDC yang diselenggarakan di Bali, Indonesia pada tahun 2017 memiliki sebuah aplikasi bernama WSDC 2017 Bali yang dikembangkan oleh PT DNArtworks menggunakan \textit{framework} Ionic 3 untuk menunjang acara tersebut. Terdapat beberapa fungsi penting di dalam aplikasi ini, diantaranya adalah jadwal untuk kegiatan peserta, berita tentang acara WSDC yang sedang berlangsung, pemberitahuan mengenai kegiatan acara kepada peserta, informasi lokasi dan penunjuk arah ke lokasi kegiatan acara yang sedang berlangsung, dan notifikasi untuk peserta. 

Aplikasi WSDC 2017 Bali yang dibangun pada tahun 2017 oleh PT DNArtworks menggunakan Ionic versi 3. Sedangkan Ionic versi 3 saat ini sudah tidak mendapat pembaruan lagi. Saat ini Ionic semakin berkembang dan sudah mencapai Ionic versi 5. Maka dari itu, pada skripsi ini akan dibuat sebuah aplikasi pembaruan dari aplikasi WSDC 2017 Bali saat ini, dengan menggunakan \textit{framework} Ionic versi 5. \textit{Framework} yang lebih baru memungkinkan perawatan yang lebih efisien, serta dukungan teknologi yang lebih terbarukan.

\section{Rumusan Masalah}
\label{sec:rumusan}
Rumusan masalah yang akan dibahas pada skripsi ini adalah sebagai berikut :
\begin{enumerate}
	\item Bagaimana melakukan migrasi aplikasi {\it android} WSDC ke {\it framework} Ionic versi 5?
\end{enumerate}


\section{Tujuan}
\label{sec:tujuan}
Tujuan yang ingin dicapai dari penulisan skripsi ini adalah sebagai berikut :
\begin{enumerate}
	\item Melakukan migrasi aplikasi {\it android} WSDC ke {\it framework} Ionic versi 5?
\end{enumerate}

\section{Batasan Masalah}
\label{sec:batasan}
Dalam skripsi ini dibuat batasan-batasan masalah dalam pembuatan perangkat lunak.  Batasan-batasan masalah yang ditetapkan adalah sebagai berikut:

\begin{enumerate}
    \item Aplikasi ini tidak akan memiliki fitur notifikasi, karena acara WSDC 2017 Bali sudah selesai dan tidak diperlukan kembali fitur notifikasi.
    \item Aplikasi hanya akan berjalan pada \textit{platform mobile} berbasis android. 
\end{enumerate}


\section{Metodologi}
\label{sec:metlit}

Langkah-langkah yang dilakukan dalam skripsi ini adalah:

\begin{enumerate}
	\item Melakukan studi mengenai {\it framework} Ionic versi 3 dan versi 5.
	\item Menganalisis aplikasi WSDC 2017 Bali.
	\item Mempelajari bagaimana cara melakukan migrasi Ionic versi 3 ke versi 5.
	\item Mendesain kelas aplikasi.
	\item Membangun aplikasi WSDC dengan {\it framework} Ionic versi 5. 
	\item Melakukan pengujian dan eksperimen.
	\item Menulis dokumen skripsi.
\end{enumerate}


\section{Sistematika Pembahasan}
\label{sec:sispem}

Sistematika penulisan setiap bab pada skripsi ini adalah sebagai berikut:
\begin{enumerate}
	\item Bab Pendahuluan \\
	Bab 1 berisi latar belakang, rumusan masalah, tujuan, batasan masalah, metodologi, dan sistematika pembahasan yang digunakan untuk menyusun skripsi ini.
	\item Bab Dasar Teori \\
	Bab 2 berisi teori-teori yang digunakan dalam pembuatan skripsi ini. Teori-teori tersebut yaitu WSDC, Angular, Ionic, Capacitor, Cordova, UI Components, dan Migrasi Ionic.
	\item Bab Analisis \\
	Bab 3 berisi analisis yang dilakukan pada skripsi ini, meliputi analisis sistem kini, analisis kebutuhan aplikasi WSDC 2017 Bali yang akan dibangun, serta permasalahan pembangunan sistem usulan.
	\item Bab Perancangan \\
	Bab 4 berisi perancangan aplikasi.
	\item Bab Implementasi dan Pengujian \\
	Bab 5 berisi implementasi dan pengujian aplikasi.
	\item Bab Kesimpulan dan Saran
	Bab 6 berisi kesimpulan dari hasil pembangunan aplikasi beserta saran untuk pengembangan selanjutnya.
	\end{enumerate}