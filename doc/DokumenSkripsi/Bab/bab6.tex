%versi 3 (22-07-2020)
\chapter{Kesimpulan dan Saran}
\label{chap:kesimpulanSaran}

\section{Kesimpulan} 
\label{sec:kesimpulan}

Dari hasil pembangunan aplikasi WSDC 2017 Bali menggunakan Ionic 6, didapatkan kesimpulan sebagai berikut:

\begin{enumerate}
	\item Telah berhasil melakukan pembaruan aplikasi WSDC 2017 Bali dengan Ionic Framework versi 6 yang sebelumnya menggunakan Ionic Framework versi 3.
	\item Aplikasi WSDC 2017 Bali telah dapat dijalankan pada perangkat dengan sistem operasi Android.
\end{enumerate}

\section{Saran} 
\label{sec:saran}

Dari hasil penelitian dan pengujian termasuk dengan pengujian terhadap responden, berikut ini merupakan beberapa saran untuk pengembangan lebih lanjut:

\begin{enumerate}
%	\item Dapat memilih font yang lebih menarik dan nyaman untuk dibaca.
%	\item Dapat memperbaiki halaman \textit{Result} agar memiliki yang tampilan lebih menarik.
%	\item Dapat mengecilkan ukuran aplikasi.
	\item Pada halaman \textit{venues} dimana tulisan headline dari list map dijauhkan sedikit dari tulisan body karena terlalu dekat.
	%\item Dapat lebih konsisten terkait dengan tampilan seperti pada halaman Venues Maps, yaitu spasi antar baris yang terkadang terlalu dekat, dan terlalu jauh.
	\item Dapat mengecilkan ukuran teks yang terlalu besar ketika pengaturan teks di ponsel diperbesar.
	\item Karena acara WSDC 2017 Bali sudah selesai, dapat dibuat untuk acara lainnya yang masih berjalan.
\end{enumerate}