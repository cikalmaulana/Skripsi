%versi 3 (22-07-2020)
\chapter{Kesimpulan dan Saran}
\label{chap:kesimpulanSaran}

\section{Kesimpulan} 
\label{sec:kesimpulan}

Dari hasil pembangunan aplikasi WSDC 2017 Bali menggunakan Ionic 6, didapatkan kesimpulan sebagai berikut:

\begin{enumerate}
	\item Telah berhasil melakukan pembaruan aplikasi WSDC 2017 Bali dengan Ionic Framework versi 6 yang sebelumnya menggunakan Ionic Framework versi 3.
	\item Aplikasi WSDC 2017 Bali telah dapat dijalankan pada perangkat dengan sistem operasi Android~~\footnote{Selain dijalankan pada perangkat Android, aplikasi WSDC 2017 Bali dengan Ionic 6 juga telah berhasi diuji dan dijalankan pada perangkat iOS. Pengujian tersebut dilakukan oleh dosen pembimbing, dan dijalankan terpisah dari dokumen ini.}.
\end{enumerate}

\section{Saran} 
\label{sec:saran}

Dari hasil penelitian dan pengujian termasuk dengan pengujian terhadap responden, berikut ini merupakan beberapa saran untuk pengembangan lebih lanjut yang berkaitan dengan tampilan aplikasi:

\begin{enumerate}
	\item Dapat memilih font yang lebih menarik dan nyaman untuk dibaca.
	\item Dapat memperbaiki halaman \textit{Result} agar memiliki yang tampilan lebih menarik.
	\item Dapat mengecilkan ukuran aplikasi.
	\item Pada halaman \textit{venues} dimana tulisan headline dari list map dijauhkan sedikit dari tulisan body karena terlalu dekat.
\end{enumerate}